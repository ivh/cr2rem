\section{Static Calibration Data}
\label{CALDATA}

In the following all raw and product IIINSTRUMENT data frames are listed,
together with the keywords used for their classification and 
correct association. The indicated {\it DO category} is a 
label assigned to any data type after it has been classified,
which is then used to identify
the frames listed in the {\it Set of Frames} (see Section
\ref{SOF}, page \pageref{SOF}).

\subsection{Photometric table} 

The optional determination of the frame magnitude 
zeropoint from the table of detected standard stars (see ahead) 
would require to specify in the input SOF a photometric table.
The photometric table simply holds the necessary parameters
for the magnitude zeropoint computation, as listed in Table \ref{PHOTO}.
The standard photometric tables in the calibration directories are
named \ {\tt ipc\_}$f${\tt .}$q${\tt .tfits} \ (where $f$ is the filter
name and $q$ the quadrant number).

\begin{table}[h]
  \begin{center}
    \begin{tabular}{|l|c|l|}
    \hline
      {\bf Keyword} & {\bf Example} & {\bf Explanation} \\
    \hline
      {\tt PRO MAG ZERO} & 28.15 & Expected magnitude zeropoint  \\
      {\tt PRO EXTINCT}  &  0.25 & Atmospheric extinction coefficient  \\
      {\tt PRO COLTERM}  &  0.01 & Correction for star colour  \\
      {\tt PRO COLOUR}   & 'B-V' & Colour system used  \\
      {\tt PRO MAGZERO RMS} & 0.05 & Error on expected zeropoint  \\
      {\tt PRO EXTINCT RMS} & 0.00 & Error on extinction coefficient  \\
      {\tt PRO COLTERM RMS} & 0.00 & Error on colour term  \\
    \hline
    \end{tabular}
    \caption{\it Photometric table entries.}
    \label{PHOTO}
  \end{center}
\end{table}



\subsection{Photometric catalog} 

The photometric catalog currently used can be 
found in the directory  \ {\tt \$PIPE\_HOME/vimos/ima/cal}, 
\ in the file \ {\tt phstd\_stetson.tfits} \ (see Table \ref{STETSON}). 
This table includes the photometric stars from the Stetson's fields 
(see \ {\tt http://cadcwww.dao.nrc.ca/standards}); Landolt's stars 
(Landolt 1992, AJ 104, 340) that can be found in the Stetson's fields 
are also included, to permit the determination of zeropoints also in 
the U band.

\begin{table}[h]
  \begin{center}
    \begin{tabular}{|l|l|}
    \hline
      {\bf Column name} & {\bf Explanation} \\
    \hline
      {\tt ID}       & Star identification string \\
      {\tt RA}       & RA of star \\
      {\tt DEC}      & Dec of star \\
      {\tt MAG\_U}   & U magnitude of star \\
      {\tt MAG\_B}   & B magnitude of star \\
      {\tt MAG\_V}   & V magnitude of star \\
      {\tt MAG\_R}   & R magnitude of star \\
      {\tt MAG\_I}   & I magnitude of star \\
    \hline
    \end{tabular}
    \caption{\it Photometric catalog entries.}
    \label{STETSON}
  \end{center}
\end{table}

\newpage


\newpage

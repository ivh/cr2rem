\section{\label{COOK}Quick start}


This section describes the most immediate usage of the IIINSTRUMENT pipeline 
recipes.


\subsection{IIINSTRUMENT pipeline recipes}
\label{RECIPES}
The current IIINSTRUMENT pipeline is based on a set of 8 stand-alone recipes 
involved in the data reduction cascade:

\begin{description}
\item {\bf si\_rec\_detlin} to evaluate the detector non linearity and 
generate a corresponding non linear pixel map.
\item {\bf si\_rec\_mdark} to create a master dark and a hot-pixel map. 
\item {\bf si\_rec\_mflat} to create a master flat and a map of pixels 
which have intensities greater than a given threshold.
\item {\bf si\_rec\_distortion} to compute the optical distortions and 
slitlets distances. 
\item {\bf si\_rec\_wavecal} for wavelength calibration.
\item {\bf si\_rec\_psf} for PSF standard data reduction.
\item {\bf si\_rec\_stdstar} for STD standard data reduction.
\item {\bf si\_rec\_objnod} for science standard data reduction.
\end{description}

\subsection{An introduction to Gasgano and EsoRex}
\label{LAUNCH}

Before being able to call pipeline recipes on a set of data, the data 
must be opportunely classified, and associated with the appropriate
calibrations. The \ {\it Data Classification} \ consists of tasks
such as: "What kind of
data am I?", {\it e.g.}, BIAS, "to which group do I belong?",
{\it e.g.}, to a particular Observation Block or template.
{\it Data Association} \ is the process of selecting appropriate
calibration
data for the reduction of a set of raw science frames. Typically, a set
of frames can be associated if they share a number of properties, such as
instrument and detector configuration. As all the required information is
stored in the FITS headers, data association is based on a set of
keywords (called "association keywords") and is specific to each type of
calibration.

The process of data classification and association is 
known as data organisation. The {\it DO Category} is the 
label assigned to a data type as a result of data classification.

An instrument pipeline consists of a set of data processing modules that
can be called from different host applications, either from the command
line with \ {\it Esorex}, \ from the automatic data management tools 
available at Paranal, or from the graphical \ {\it Gasgano} \ tool.

{\it Gasgano} \ is a data management tool that simplifies the data 
organisation process, offering automatic data classification and making
the data association easier ({\it even if automatic association of frames is 
not yet provided}). {\it Gasgano} \ determines the classification of a file
by applying an instrument specific rule, while users must provide this
information to the recipes when they are executed manually using
\ {\it Esorex} \ from the command line. In addition, 
\ {\it Gasgano} \ allows the user to execute 
directly the pipeline recipes on a set of selected files. 

\subsubsection{Using Gasgano}
\label{GASGANO}


 
To get familiar with the IIINSTRUMENT pipeline recipes and their usage, it 
is advisable to begin with \ {\it Gasgano}, \ because it provides
a complete graphic interface for data browsing, classification and 
association, and offers several other utilities such as easy access 
to recipes documentation and preferred data display tools.

{\it Gasgano} \ can be started from the system prompt in the following way:

\begin{verbatim}
   gasgano &
\end{verbatim}


The \ {\it Gasgano} \ main window will appear. On Figure \ref{GAS_MAIN}
(next page), a view on a set of IIINSTRUMENT IFU data is shown as an 
example. \ {\it Gasgano} \ can 
be pointed to the directories where the data to be handled are located
using the navigation panels accessible via the \ {\it Add/Remove Files} \
entry of the \ {\it File} \ menu (shown on the upper left of the figure). 

The data are hierarchically organised as preferred by the user.
After each file name are shown the classification, 
the instrument setup id (which indicates the band), the instrument 
pre-optic (which indicates the camera setting), the template exposure 
number and the number of exposures in the template, and the value 
of the DPR.TYPE. 

More information about a single frame
can be obtained by clicking on its name: the corresponding
FITS file header will be displayed on the bottom panel, where
specific keywords can be opportunely filtered and searched.
Images and tables may be easily displayed using the 
viewers specified in the appropriate \ {\it Preferences} \ fields.

% The file names corresponding to raw data ({\it i.e.}, those produced 
% by the instrument) appear in blue, while the names of the processed 
% data and of the static calibration tables appear in red. In Figure 
% \ref{GAS_MAIN} is clearly visible the master bias product, 
% \ {\tt master\_bias1.fits}, \ already derived from the raw bias 
% exposures belonging to the same sub-group: in this way the association
% of a product frame with the corresponding raw files is made evident
% by the \ {\it Gasgano} \ interface.


Frames can be selected from the main window for being processed
by the appropriate recipe: on Figure \ref{GAS_SEL}, the standard star frame 
and a sky frame, already produced master bad pixel map and 
master flat field frames, together with distortion and slitlet distance tables,
and the necessary static calibration tables, are all selected 
and sent to the \ {\it si\_rec\_stdstar} \ recipe. This will
open a \ {\it Gasgano} \ recipe execution window (see Figure
\ref{GAS_REC}), having all the specified files listed in its 
\ {\it Input Frames} \ panel.

Help about the recipe may be obtained from the \ {\it Help} \ menu.
Before launching the recipe, its configuration may be
opportunely modified on the \ {\it Parameters} \ panel (on top). 
The window contents might be saved for later use
by selecting the \ {\it Save Current Settings} \ entry from the 
\ {\it File} \ menu, as shown in figure.

At this point the recipe can be launched by pressing the \ {\it Execute} \
button. Messages from the
running recipe will appear on the \ {\it Log Messages} panel at
bottom, and in case of successful completion the products will be listed
on the \ {\it Output Frames} \ panel, where they can be easily viewed
and located back on the Gasgano main window.

Please refer to the \ {\it Gasgano User's Manual} [7] for a more complete
description of the \ {\it Gasgano} \ interface.


\clearpage
\begin{figure}[h]
\begin{center}
%\includegraphics[width=23truecm]{gasgano1.eps}
\end{center}
\caption{\it The Gasgano main window.}
\label{GAS_MAIN}
\end{figure}



\begin{figure}[h]
\begin{center}
%\includegraphics[width=23truecm]{gasgano2.eps}
\end{center}
\caption{\it Selecting files to be processed by a IIINSTRUMENT pipeline recipe.}
\label{GAS_SEL}
\end{figure}

\begin{figure}[h]
\begin{center}
%\includegraphics[width=23truecm]{gasgano3.eps}
\end{center}
\caption{\it The Gasgano recipe execution window.}
\label{GAS_REC}
\end{figure}

\clearpage

%%%



\subsubsection{Using EsoRex}
\label{ESOREX}

{\it EsoRex} \ is a command line utility for running pipeline recipes. 
It may be embedded by users into data reduction scripts for the automation 
of processing tasks. On the other side, \ {\it EsoRex} \ doesn't offer 
all the facilities available with \ {\it Gasgano}, \ and the user must 
classify and associate the data using the information contained in 
the FITS header keywords (see Section \ref{DATA:GEN}, 
page \pageref{DATA:GEN}). 
The user should also take care of defining the input set-of-frames and 
the appropriate configuration parameters for each recipe run:

\begin{description}

\item [The set-of-frames:]
\label{SOF}
Each pipeline recipe is run on a set of input FITS data files. 
When using \ {\it EsoRex} \ the 
filenames must be listed together with their DO category in an ASCII file, 
the \ {\it set-of-frames} \ (SOF), that is required when launching a recipe.
\footnote{The set-of-frames corresponds to the \ {\it Input Frames} \ panel
of the \ {\it Gasgano} \ recipe execution window (see Figure \ref{GAS_REC},
page \pageref{GAS_REC}).}
% SOF files containing the frames selected by the user are automatically 
% created by \ {\it Gasgano} [7], while they need to be created in some
% way by the user working with \ {\it EsoRex}.

Here is an example of SOF, valid for the \ {\it si\_rec\_wavecal} \ recipe
\footnote{We list the file SLIT\_POS\_H\_250.fits as an input file, as,
for robustness, we suggest the user to set the parameter 
{\bf slitpos\_bootstrap\_switch} to FALSE. A different setting would allow
to reduce the data without including the SLIT\_POS table}:
\begin{verbatim}
    /file_path/SINFO.2004-08-14T10:20:56.497.fits  WAVE_LAMP
    /file_path/SINFO.2004-08-14T10:22:44.285.fits  WAVE_LAMP
    /file_path/xenon.tfits                         REF_LINE_ARC
    /file_path/MASTER_BP_MAP_H_250.fits            MASTER_BP_MAP
    /file_path/MASTER_LAMP_FLAT_H_250.fits         MASTER_FLAT_LAMP
    /file_path/DISTORTION_H.fits                   DISTORTION
    /file_path/drs_setup_wave.tfits                DRS_SETUP_WAVE
    /file_path/SLIT_POS_H_250.tfits                SLIT_POS
\end{verbatim}

It contains for each input frame the full path file name and its 
DO category.
The pipeline recipe will access the listed files when required 
by the reduction algorithm.

Note that the IIINSTRUMENT pipeline recipes do not verify in any way the 
correctness of the {\it DO Category} specified by the user in the SOF. 
The reason of this lack of control is that the IIINSTRUMENT recipes are just 
the DRS component of the complete pipeline running on Paranal, where 
the task of data classification and association is carried out by 
separate applications. Moreover, using \ {\it Gasgano} \ as
an interface to the pipeline recipes will always ensure a correct
classification of all the data frames, assigning the appropriate 
DO category to each one of them (see Section \ref{GASGANO}, page
\pageref{GASGANO}).

A recipe handling an incorrect SOF may stop or display unclear 
error messages at best. In the worst cases, the recipe 
would apparently run without any problem, producing results that may 
look reasonable, but are actually flawed.

\item [EsoRex syntax:]

The basic syntax to use ESOREX is the following:

{\bf esorex [esorex\_options] recipe\_name [recipe\_options] set\_of\_frames}

To get more information on how to customise ESOREX (see also [7]) run the command:

{\bf esorex -\,-help}

To generate a configuration file esorex.rc in the directory \${HOME}/.esorex run the command:

{\bf esorex -\,-create-config}

A list of all available recipes, each with a one-line description, can be 
obtained using the command:

{\bf esorex -\,-recipes}

All recipe parameters (aliases) and their default values can be displayed by 
the command

{\bf esorex -\,-params recipe\_name}

To get a brief description of each parameter meaning execute the command:

{\bf esorex -\,-help recipe\_name}

To get more details about the given recipe give the command at the shell 
prompt:

{\bf esorex -\,-man-page recipe\_name}


\item [Recipe configuration:]

Each pipeline recipe may be assigned an \ {\it EsoRex} \ configuration 
file, containing the default values of the parameters related to that 
recipe.\footnote{The \ {\it EsoRex} \ recipe configuration file 
corresponds to the \ {\it Parameters} \ panel of the \ {\it Gasgano} \ 
recipe execution window (see Figure \ref{GAS_REC}, page \pageref{GAS_REC}).}
The configuration files are normally generated in the directory 
\ {\tt \$HOME/.esorex}, 
%(as will be shown in the subsequent examples),
and have the same name as the recipe to which they are related, 
with the filename extension \ {\tt .rc}. For instance, the recipe 
\ {\it si\_rec\_wavecal} \ has its \ {\it EsoRex} \ generated configuration 
file named \ {\tt si\_rec\_wavecal.rc}, and is generated with the command:

{\bf esorex -\,-create-config si\_rec\_wavecal}

The definition of one parameter of a recipe may look like this:
\begin{verbatim}
# --stack-warpfix_kernel
# Warpfix kernel:  (tanh | sinc | sinc2 | lanczos | hamming | hann)
sinfoni.stacked.warpfix_kernel=tanh
\end{verbatim}


In this example, the parameter \ {\tt sinfoni.stacked.warpfix\_kernel} 
\ is set to the value \ {\tt tanh}. \ In the configuration file
generated by \ {\it EsoRex}, \ one or more comment lines are added
containing information about the possible values of the parameter, and
an alias that could be used as a command line option.  

The recipes provided by the IIINSTRUMENT pipeline are designed to implement 
a cascade of macro data reduction steps, each controlled by 
its own parameters.
For this reason and to prevent parameter name clashes we specify as 
parameter prefix not only the instrument name but also the name of the 
step they refer to. Shorter parameter aliases are made available for 
use on the command line.

The command

{\bf esorex -\,-create-config recipe\_name}

generates a default configuration file {\bf recipe\_name.rc} in the directory 
{\bf \${HOME}/.esorex}\footnote{If a number of recipe parameters are 
specified on the command line, the given values will be used in the
created configuration file.}.

A recipe configuration file different from the default one can be 
specified on the command line:

{\bf esorex -\,-recipe-config=my\_alternative\_recipe\_config}


Recipe parameters are provided in section \ref{pipeline_recipe_interfaces} 
and their role is described in Section \ref{algorithms}.

More than one configuration file may be maintained for the same recipe
but, in order to be used, a configuration file not located under 
\ {\tt \$HOME/.esorex}, \ or having a name different from the recipe 
name, should be explicitly specified when launching a recipe.


\item [Recipe execution:]


A recipe can be run by specifying its name to 
\ {\it EsoRex}, together with the name of a set-of-frames.
For instance, the
following command line would be used to run the recipe \ {\it si\_rec\_wavecal}
\ for processing the files specified in the set-of-frames 
\ {\tt si\_rec\_wavecal.sof}:

   {\bf esorex si\_rec\_wavecal si\_rec\_wavecal.sof}

The recipe parameters can be modifyed either by editing directly the 
used configuration file, or by specifying new parameter values
on the command line using the command line options defined for this
purpose. Such command line options should be inserted after the
recipe name and before the SOF name, and they will supersede the
system defaults and/or the configuration file settings. For instance, 
to set the
\ {\it si\_rec\_wavecal} \ recipe \ {\it wcal-pixel\_tol} \ parameter to 
\ {\tt 3.0},
\ the following should be typed:


   {\bf esorex si\_rec\_wavecal -\,-wcal-pixel\_tol=3.0 si\_rec\_wavecal.sof}


\end{description}

For more information on \ {\it EsoRex}, \ see 
\ {\tt http://www.eso.org/cpl/esorex.html}.


\subsection{Example of data reduction using EsoRex}
\label{EXIFU}

A simple, typical data reduction procedure is described 
here.\footnote{The procedure using \ {\it Gasgano} 
\ is conceptually identical.}


We suggest the user to organize his data per type, 
observed band and camera setting. Dark frames may be
grouped per detector DIT, frames to compute distortion and frames to
compute detector non linearities may be organized
per observed band. The detector DIT is given by the value of the
FITS keyword DET DIT\footnote{We omit here the prefix HIERARCH ESO}. 
The observed band is indicated by the value of the FITS keyword 
INS SETUP ID. The camera setting is indicated by the value of
INS OPTI1 NAME. In the examples below we suppose the user has data 
acquired in band K and with the 100 mas pre-optic setting, and DIT=600.
In the following examples {\tt /path\_raw/} indicates the full path to the
source tree directory containing raw data.

Dark Frames: those frames are characterized by DPR.TYPE='DARK',

\begin{verbatim}
/path_raw/DARK/600/SINFO.2004-08-23T09:36:12.316.fits DARK
/path_raw/DARK/600/SINFO.2004-08-23T09:51:59.824.fits DARK
/path_raw/DARK/600/SINFO.2004-08-23T10:07:40.760.fits DARK
\end{verbatim}


Detector linearity flat field frames: those frames are caracterized 
by DPR.TYPE='LINEARITY,LAMP'

\begin{verbatim}
/path_raw/LINEARITY/K/SINFO.2005-02-26T20:09:50.882.fits LINEARITY_LAMP
/path_raw/LINEARITY/K/SINFO.2005-02-26T20:10:07.455.fits LINEARITY_LAMP
/path_raw/LINEARITY/K/SINFO.2005-02-26T20:10:23.047.fits LINEARITY_LAMP
/path_raw/LINEARITY/K/SINFO.2005-02-26T20:10:38.240.fits LINEARITY_LAMP
/path_raw/LINEARITY/K/SINFO.2005-02-26T20:10:56.403.fits LINEARITY_LAMP
/path_raw/LINEARITY/K/SINFO.2005-02-26T20:11:31.128.fits LINEARITY_LAMP
/path_raw/LINEARITY/K/SINFO.2005-02-26T20:11:59.183.fits LINEARITY_LAMP
/path_raw/LINEARITY/K/SINFO.2005-02-26T20:12:27.247.fits LINEARITY_LAMP
\end{verbatim}




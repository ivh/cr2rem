\section{Algorithms}
\label{sec:ALGORITHMS}

In this section the data reduction procedures applied by the 23 pipeline
recipes currently in use (see Section \ref{RECIPES}) are described 
in some detail. Common algorithms, as cosmic rays removal
or bad pixel cleaning, are described separately.

\subsection{General Algorthms}

\subsubsection{Bad pixel cleaning}
\label{sec:ABADPIX}

Bad pixel cleaning consists of replacing any bad pixel value
with an estimate based on a set of surrounding \ {\it good} 
\ pixel values. This operation is generally applied to science 
product frames, having little or no sense when applied to master 
calibration products. Nevertheless all the \pipename\, pipeline recipes 
allow bad pixel cleaning on any product frame, for debug reasons 
or for any other purpose that may be appropriate.

The routine currently used by the \pipename\, pipeline recipes performs 
a bad pixel correction based on the content of a given bad pixel 
table \ ({\tt CCD\_TABLE}). \ If the number of bad pixels is more
than 15\% of the total number of CCD pixels, the correction is not
applied.

\putgraph{14}{badpix.png}{IBADPIX}{Good pixels to be used in the estimate of
a given bad pixel are searched along the indicated directions.}

Any bad pixel is given a new value, computed as follow: the closest
good pixels along the vertical, the horizontal, and the two diagonal 
directions are found (see Figure \ref{fig:IBADPIX}). This search is
done within a distance of 100 pixels. If no good pixel is found
within this range, then the bad pixel is not corrected.
All the good pixels found within range will 
be used to compute the bad pixel value.

For each of the four fundamental directions, an estimate of the
considered bad pixel can generally be obtained. 
If two good pixel values are available for a given direction,
the estimate is their linear interpolation at the bad pixel position.
If just one good pixel value is available for a given direction, 
then the value itself will be the estimate of the bad pixel value.
No estimate can be obtained from directions where no good pixel
was found.

If the available number of estimates is greater than 1, the
bad pixel value is taken as the median of the estimates (defining
the median of an even number of values as the mean of the two
central values), otherwise it is simply set to the single estimate 
available.

\subsubsection{Cosmic rays removal}
\label{sec:ACOSMIC}

The core of a cosmic rays removal procedure is to determine what is
and what is {\it not} a cosmic ray. The algorithm used for this
purpose by the \pipename\, pipeline recipes is the same applied by the MIDAS 
command \ {\tt FILTER/COSMIC}, \ with some extensions.

Initially all pixels having an abnormal excess with respect
to the local noise level are flagged as possibly belonging to a
cosmic ray event (that typically would involve a group of contiguous 
pixels).  A candidate is selected at any pixel \ $(x,y)$ \ having a value 
\ $F(x,y)$ \ exceeding a given threshold. This threshold, expressed in
units of noise sigma, is specified by the recipe parameter 
\ {\it CosmicsThreshold}. \ A value 4.0 gives typically good results.
The theoretical
noise \ $N(x,y)$ \ of the image at any given pixel position \ $(x,y)$ \ is 
estimated in ADU as 

$$
          N(x,y) = \sqrt{r^2 + {M(x,y) \over g}}
$$

%%% PLEASE NOTE %%% %%% %%% %%% %%% %%% %%% %%% %%% %%% %%% %%% %%% %%% %%%
%%% Apparently a good number of people reading this manual came to think
%%% that this formula is wrong. Indeed it is a bit counter-intuitive,
%%% so I add an explanation here (I don't want to include it in the manual
%%% because it is quite a common formula used in CCD data reduction, that
%%% is very well known among [almost] all DRS developers.
%%%
%%% The formula is derived in the following way:
%%% 
%%% 1) Let be M the signal level, and r the read-out-noise, both in ADU,
%%%    and g the conversion factor between ADU and electrons. Then the
%%%    signal level in electrons is given by gM, and the read-out-noise
%%%    in electrons is given by gr.
%%%
%%% 2) Assuming poissonian statistics, the noise (in electrons) associated
%%%    to the signal gM is given by sqrt(gM).
%%%
%%% 3) The total noise (in electrons) can then be computed as the
%%%    geometrical average of the noise components:
%%%
%%%                sqrt{ [sqrt(gM)]^2 + (gr)^2 }
%%%
%%%    that is
%%%
%%%                sqrt{ gM + (gr)^2 }
%%%
%%% 4) To obtain the total noise in ADU the above result should be divided
%%%    by the gain factor g, obtaining
%%%
%%%                sqrt{ M/g + r^2 }
%%%
%%%    that is the used formula shown above.
%%%
%%% %%% %%% %%% %%% %%% %%% %%% %%% %%% %%% %%% %%% %%% %%% %%% %%% %%% %%%

where \ $M(x,y)$ \ is the median value of the 8 pixels 
surrounding the \ $(x,y)$ \ position 
and \ $r$ \ is the read-out-noise, both in ADU, 
and \ $g$ \ is the gain
factor in \ $e^-/ADU$. \ Then a pixel \ $(x,y)$ \ is taken as a cosmic 
ray candidate if
$$
             F(x,y) > k \cdot N(x,y)
$$
with \ $k$ \ the number of noise sigmas used in thresholding.

After this step is completed, all the groups of contiguous cosmic
rays candidates are identified. For each group, the position of 
its maximum pixel value is determined, and the mean 
\ $\overline{F}_8$ \ of its 
8 surrounding pixels is computed. A given group will be taken as a
cosmic ray event if it fulfils the condition
$$
             F_{max} - S > R \cdot (\overline{F}_8 - S)
$$
where \ $F_{max}$ \ is the maximum pixel value within the considered group,
\ $S$ \ the fundamental background level (corresponding to the sky level
in imaging science exposures), and \ $R$ \ is a shape parameter for 
discriminating between objects and cosmic rays. The ratio \ $R$ \ is 
specified by the recipe parameter \ {\it CosmicsRatio}. 
\ A value of 2.0 gives typically good results.

Once all the pixels affected by cosmic ray events has been located
and listed in a cosmic ray events table, their values are interpolated 
using the procedure described in Section \ref{sec:ABADPIX}. If a bad pixel 
table is also given to a recipe, then the bad pixels are avoided in the 
interpolation procedure.

\subsection{Recipes Algorithms} 

\subsubsection{gi\_rec\_bias}
\label{sec:ABIAS}

Removing the bias from any raw frame is a relatively simple
process, but not simple enough to avoid a description on its own.

A master bias frame ({\tt MASTER\_BIAS}) is used to remove
the bias level (and, if present, the fixed-pattern-noise related
to the bias) from a raw frame. Typically a master calibration
is produced with its overscan regions trimmed, and if this is the
case with the master bias used then its missing overscan regions
are extrapolated by repeating the signal contained in its border
regions with equal size.

The master bias is subtracted from the raw data frame, whose
overscan regions are then trimmed away. Optionally (when 
the \ {\it BiasMethod} \ recipe parameter is set to {\it ``Zmaster''})
the residual signal in the overscan regions is averaged along
the \ $X$ \ CCD coordinate, and the obtained mean \ $Y$ \ values are
modeled with a second order polynomial fitting. This model is 
then subtracted from the rest of the image.

\subsubsection{gi\_rec\_dark}
\label{sec:ADARK}

Subtracting the dark current component from any raw frame consists 
of multiplying an input master dark frame by the exposure time 
(in seconds) of the frame to be corrected, and then subtract such
rescaled dark frame from it. The dark level is quite low for IIINSTRUMENT CCDs
(about 5 $e^{-} \cdot $ h$^{-1} \cdot $ pixel$^{-1}$), so this operation
would be in most cases superfluous.

\subsubsection{gi\_rec\_flat}
\label{sec:AFLATFIELDING}
\label{sec:AFLAT}
\label{sec:AFLATSKY}

The flat field correction merely consists of dividing the frame 
to be corrected by a given master sky flat field frame produced 
by the recipe \ {\it vmimflatsky} \ for direct imaging observation
(see Section \ref{sec:AFLATSKY}, page \pageref{sec:AFLATSKY}), or produced 
by the recipe \ {\it vmspflat} \ for MOS observations (see Section 
\ref{sec:AFLAT}, page \pageref{sec:AFLAT}).

\newpage


\section{Instrument Data Description}
\label{DATA}

IIINSTRUMENT data can be separated into {\it raw} frames and {\it product} frames.
Raw frames are the unprocessed output of the IIINSTRUMENT instrument observations,
while product frames are either the result of the IIINSTRUMENT pipeline processing 
(as reduced frames, master calibration frames, etc.), or are outsourced (as 
standard stars catalogs, lists of grism characteristics, etc.).

Any raw or product frame can be classified on the basis of a set of keywords 
read from its header. Data classification is typically carried out by the 
DO or by \ {\it Gasgano} [7], that apply the same set of classification
rules. The 
association of a raw frame with calibration data ($e.g.$, of a science 
frame with a master bias frame) can be obtained by matching the values 
of a different set of header keywords.

Each kind of \ raw \ frame is typically associated to a single IIINSTRUMENT
pipeline recipe, {\it i.e.}, the recipe assigned to the reduction of that
specific frame type. In the pipeline environment this recipe would 
be launched automatically. In some cases two recipes are assigned,
one meant for the reduction of a single frame of that type, and the
other for the reduction of a \ {\it stack} \ of frames of the same 
type, as happens in the case of jittered science observations. 

A \ product \ frame may be input to more than one IIINSTRUMENT pipeline recipe, 
but it may be created by just one pipeline recipe (with the same
exceptions mentioned above). In the automatic pipeline environment a 
product data frame alone wouldn't trigger the launch of any recipe.

In the following all raw and product IIINSTRUMENT data frames are listed,
together with the keywords used for their classification and 
correct association. The indicated {\it DO category} is a 
label assigned to any data type after it has been classified,
which is then used to identify
the frames listed in the {\it Set of Frames} (see Section
\ref{SOF}, page \pageref{SOF}).

\label{DATA:RAW}

Raw frames can be distinguished in {\it general} frames, {\it direct imaging}
frames, {\it MOS} frames and {\it IFU} frames. Their intended use is
implicitly defined by the assigned recipe.

\subsection{General frames}
\label{DATA:GEN}

These are data that are in principle independent of the instrument mode 
(direct imaging, MOS, or IFU), as is the case for bias and dark exposures. 
The keyword \ {\tt ESO INS MODE} \ is set accordingly to 'IMG' for direct 
imaging frames, and to 'MOS' for any calibration associated to spectroscopy 
(either MOS or IFU), to indicate the intended use for the data.


{\small \begin{itemize}

\item {\bf Bias:}

DO category: {\tt BIAS} \\
Processed by: {\tt vmbias}

\begin{tabbing}
{\tt x} \= {\tt xxxxxxxxxxxxxxxxxxxxxxxxxxxxxxxxx} \= {\tt xxxxxxxxxxxxxxxxxxx} \= {\tt xxxxxxxxxxxx} \kill
\> Classification keywords: \> Association keywords: \> Note: \\
\> {\tt DPR CATG = CALIB} \> {\tt INS MODE} \> Instrument mode \\
\> {\tt DPR TYPE = BIAS} \> {\tt OCS CON QUAD} \> Quadrant used \\
\> {\tt DPR TECH = IMAGE} \> {\tt DET CHIP1 ID} \> Chip identification \\
\> \> {\tt DET WIN1 NY} \> No of pixels in y \\
\> \> {\tt DET WIN1 BINX} \> Binning along X \\
\> \> {\tt DET WIN1 BINY} \> Binning along Y \\
\> \> {\tt DET READ MODE} \> Readout method \\
\> \> {\tt DET READ SPEED} \> Readout speed \\
\> \> {\tt DET READ CLOCK} \> Readout clock pattern \\
\end{tabbing}

\item {\bf Dark current:}

DO category: {\tt DARK} \\
Processed by: {\tt vmdark}

\begin{tabbing}
{\tt x} \= {\tt xxxxxxxxxxxxxxxxxxxxxxxxxxxxxxxxx} \= {\tt xxxxxxxxxxxxxxxxxxx} \= {\tt xxxxxxxxxxxx} \kill
\> Classification keywords: \> Association keywords: \> Note: \\
\> {\tt DPR CATG = CALIB} \> {\tt INS MODE} \> Instrument mode \\
\> {\tt DPR TYPE = DARK} \> {\tt OCS CON QUAD} \> Quadrant used \\
\> {\tt DPR TECH = IMAGE} \> {\tt DET CHIP1 ID} \> Chip identification \\
\> \> {\tt DET WIN1 NY} \> No of pixels in y \\
\> \> {\tt DET WIN1 BINX} \> Binning along X \\
\> \> {\tt DET WIN1 BINY} \> Binning along Y \\
\> \> {\tt DET READ MODE} \> Readout method \\
\> \> {\tt DET READ SPEED} \> Readout speed \\
\> \> {\tt DET READ CLOCK} \> Readout clock pattern \\
\end{tabbing}

\item {\bf Screen flat field for gain determination and bad pixels detection:}

DO category: {\tt DETECTOR\_PROPERTIES} \\
Processed by: {\tt vmdet}

\begin{tabbing}
{\tt x} \= {\tt xxxxxxxxxxxxxxxxxxxxxxxxxxxxxxxxx} \= {\tt xxxxxxxxxxxxxxxxxxx} \= {\tt xxxxxxxxxxxx} \kill
\> Classification keywords: \> Association keywords: \> Note: \\
\> {\tt DPR CATG = CALIB} \> {\tt INS MODE} \> Instrument mode \\
\> {\tt DPR TYPE = FLAT,LAMP} \> {\tt OCS CON QUAD} \> Quadrant used \\
\> {\tt DPR TECH = IMAGE} or {\tt MOS} \> {\tt DET CHIP1 ID} \> Chip identification \\
\> {\tt TPL ID = IIINSTRUMENT\_img\_tec\_DetLin} \> {\tt DET WIN1 NY} \> No of pixels in y \\
\> or {\tt IIINSTRUMENT\_mos\_tec\_DetLin} \> {\tt DET WIN1 BINX} \> Binning along X \\
\> \> {\tt DET WIN1 BINY} \> Binning along Y \\
\> \> {\tt DET READ MODE} \> Readout method \\
\> \> {\tt DET READ SPEED} \> Readout speed \\
\> \> {\tt DET READ CLOCK} \> Readout clock pattern \\
\end{tabbing}

\end{itemize}}


\subsection{Direct imaging frames}
\label{DATA:DIR}

The direct imaging mode is used to record signal without using any grism.

{\small \begin{itemize}

\item {\bf Exposure of calibration mask:}

DO category: {\tt MASK\_TO\_CCD} \\
Processed by: {\tt vmmasktoccd}

\begin{tabbing}
{\tt x} \= {\tt xxxxxxxxxxxxxxxxxxxxxxxxxxxxxxxxx} \= {\tt xxxxxxxxxxxxxxxxxxx} \= {\tt xxxxxxxxxxxx} \kill
\> Classification keywords: \> Association keywords: \> Note: \\
\> {\tt DPR CATG = CALIB} \> {\tt INS MODE} \> Instrument mode \\
\> {\tt DPR TYPE = OTHER,LAMP} \> {\tt OCS CON QUAD} \> Quadrant used \\
\> {\tt DPR TECH = IMAGE} \> {\tt INS ADF ID} \> ADF file ID \\
\> {\tt TPL ID = IIINSTRUMENT\_img\_tec\_MaskToCcd} \> {\tt INS FILT[1-4] ID} \> Filter ID on each beam \\
 \> \> {\tt DET CHIP1 ID} \> Chip identification \\
 \> \> {\tt DET WIN1 NY} \> No of pixels in y \\
 \> \> {\tt DET WIN1 BINX} \> Binning along X \\
 \> \> {\tt DET WIN1 BINY} \> Binning along Y \\
 \> \> {\tt DET READ MODE} \> Readout method \\
 \> \> {\tt DET READ SPEED} \> Readout speed \\
 \> \> {\tt DET READ CLOCK} \> Readout clock pattern \\
\end{tabbing}

\end{itemize}}




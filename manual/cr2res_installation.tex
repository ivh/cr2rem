\section{Installation}
\label{sec:installation}

\red{\textit{Installation instructions for the different package types offered
    by ESO and build instructions for experienced users. This part will be
    filled by ESO to a large extent.}}

\subsection{System Requirements}
\label{sec:platforms}

\subsection{Installing the \instrument{} Pipeline}
\label{sec:install-howto}

For Fedora, CentOS, Scientific Linux users within ESO, you can use the RPM
installation.
For that, you need to refer to the RPM installation instructions:
http://eso.org/sci/software/pipelines/installation/rpm.html

CR2RES pipeline is not public yet, therfore you will need to setup the
repositories to a different place: 
USE 'devel' instead of 'stable' ! 

For example:

sudo dnf config-manager --add-repo=ftp://ftp.eso.org/pub/dfs/pipelines/repositories/stable/fedora/esorepo.repo

needs to be replaced by:

sudo dnf config-manager --add-repo=ftp://ftp.eso.org/pub/dfs/pipelines/repositories/devel/fedora/esorepo.repo


For any other system, or if you are outside ESO, you will need to download the installation script, and run it on your computer:

- Create an new directory and change to it
- Download http://eso.org/sci/software/pipelines/install\_pipelinekit
and change its permission to execute (chmod u+x install\_pipelinekit)
- Download the CR2RES package:
  
ftp://ftp.eso.org/pub/dfs/pipelines/kit\_devel/cr2re-kit-0.5.0-2\_svn232540\_head.tar.gz

- Execute 
    ./install\_pipelinekit cr2re-kit-0.5.0-2\_svn232540\_head.tar.gz

    The execution scipt will ask you where to install the package. I recommend to have a local installation in order to easily update at a later stage.

    Questions asked by the installation script:

    * Where should I install the software packages ? 
        Use a new directory (e.g. /home/user/cr2res\_install)

    * Where should I install the pipeline calibration files ?
        Use a subdirectory (e.g. /home/user/cr2res\_install/cdb)

- Configuration

    * esorex is installed here : /home/user/cr2res\_install/bin/esorex 
        Make sure it is in your path.

    * The CR2RES recipes are installed under : /home/user/cr2res\_install/lib/esopipes-plugins/cr2re-0.5.0
        esorex needs to be configured to know where th CRI2RES recipes are installed. This is done through the recipe-dir parameter.
        Please update :
        esorex.caller.recipe-dir=/home/user/cr2res\_install/lib/esopipes-plugins/cr2re-0.5.0
        in the esorex configuration file (usually \$HOME/.esorex/esorex.rc)

    * If you do not have any esorex configuration file, you can run
        esorex --create-config 
        to create a default one.

    * You know you are all set when running 
        esorex --recipes
        gives you the list of installed recipes.

If you want to install a new version, simply remove the directory
/home/user/cr2res\_install and start again.






\subsection{Building the \instrument{} Pipeline}
\label{sec:compile-howto}

\subsubsection{Build Requirements}
\label{sec:compile-requirements}

%%% Local Variables: 
%%% mode: latex
%%% TeX-master: t
%%% End: 

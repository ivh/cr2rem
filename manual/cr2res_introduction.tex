\section{Introduction}
\label{sec:introduction}

\subsection{Scope}
\label{sec:scope}
This document is the user manual used to process of \instrument{}
observations with the CR2RES
Pipeline Recipes, also known as the \instrument{} Data Reduction Software 
(DRS).


\subsection{Acknowledgements}
We are thankful for comments on this manual from Jonathan Smoker and Ulf
Seemann.

Special thanks also to Nikolai Piskunov and Ansgar
Wehrhahn for their extensive contributions of code
to the DRS.



\subsection{Stylistic conventions}
\label{sec:style}

Throughout this document the following stylistic conventions are used:

\begin{tabular}{lp{0.75\linewidth}}
\textbf{bold}     & in text sections for commands and other
                    user input which has to be typed as shown \\
\textit{italics}  & in the text and example sections for parts of the user
                    input which have to be replaced with real contents \\
\texttt{teletype} & in the text for FITS keywords, program names, file paths,
                    and terminal output, and as the general style for examples,
                    commands, code, etc \\
\end{tabular}

In example sections expected user input is indicated by a leading shell
prompt.

In the text \textbf{bold} and \textit{italics} may also be used to highlight
words.

\subsection{Notational Conventions}
\label{sec:notation}

The DRS for the original, pre-upgrade CRIRES (oCRIRES) used the prefix
\texttt{crires\_} for recipe names (\cite{OCIRESCOOK}). In order to avoid
confusion and because the pipeline for \instrument\ was rewritten from scratch
and cannot reduce data from oCRIRES, a new prefix was chosen: \texttt{cr2res\_}.

Hierarchical FITS keyword names, appearing in the document, are given using the
dot--notation to improve readability. This means, that the prefix ``HIERARCH
ESO'' is left out, and the spaces separating the keyword name constituents in
the actual FITS header are replaced by a single dot.

%\section{Related Documents}
%\label{sec:doc-related}

%\subsection{Applicable Documents}
%\label{sec:doc-applicable}

%\begin{tabularx}{\linewidth}{lllX}
%    {[}AD01{]} & VLT-MAN-ESO-19500-1619  & \cite{VLT1619}
%               & DFS Pipeline \& Quality Control -- User Manual \\
%\end{tabularx}

\subsection{Reference Documents}
\label{sec:doc-reference}

\begin{tabularx}{\linewidth}{llX}
  {[}RD01{]} & ESO-254264 \cite{CIRESMAN}
             & CRIRES+ User Manual \\
  {[}RD02{]} & 2021A\&A...646A..32P \cite{2021A&A...646A..32P} 
             & Optimal extraction of echelle spectra \\
  {[}RD03{]} & VLT-MAN-ESO-14200-4032 \cite{OCIRESCOOK}
             & oCRIRES Data Reduction Cookbook \\
  {[}RD04{]} &  1997MNRAS.291..658D \cite{1997MNRAS.291..658D}
             & Spectropolarimetric observations of active stars \\
  {[}RD05{]} &  2019A\&A...624A.122C \cite{2019A&A...624A.122C}
             & New wavelength calibration for echelle spectrographs using Fabry-P{\'e}rot etalons 

\end{tabularx}


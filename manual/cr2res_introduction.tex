\section{Introduction}
\label{sec:introduction}

\subsection{Scope}
\label{sec:scope}

This document defines the design of the data reduction library for the KMOS pipeline, including
all modules of the DRL to process KMOS data as well as the additional DFS tools. It provides a
technical description of the instrument modes, data formats and data processing required for
scientific observations, calibrations, and instrument monitoring tasks for KMOS. It is based on
the DRL Specification [AD02] and supersedes that document.

%\subsection{Acknowledgements}

\subsection{Stylistic conventions}
\label{sec:style}

Throughout this document the following stylistic conventions are used:

\begin{tabular}{lp{0.75\linewidth}}
\textbf{bold}     & in text sections for commands and other
                    user input which has to be typed as shown \\
\textit{italics}  & in the text and example sections for parts of the user
                    input which have to be replaced with real contents \\
\texttt{teletype} & in the text for FITS keywords, program names, file paths,
                    and terminal output, and as the general style for examples,
                    commands, code, etc \\
\end{tabular}

In example sections expected user input is indicated by a leading shell
prompt.

In the text \textbf{bold} and \textit{italics} may also be used to highlight
words.

\subsection{Notational Conventions}
\label{sec:notation}

Hierarchical FITS keyword names, appearing in the document, are given using the
dot--notation to improve readability. This means, that the prefix ``HIERARCH
ESO'' is left out, and the spaces separating the keyword name constituents in
the actual FITS header are replaced by a single dot.

\section{Related Documents}
\label{sec:doc-related}

\subsection{Applicable Documents}
\label{sec:doc-applicable}

\begin{tabularx}{\linewidth}{lllX}
  {[}AD01{]} & ESO-XXXXXX & 1.0
             & TBD \\
\end{tabularx}

\subsection{Reference Documents}
\label{sec:doc-reference}

\begin{tabularx}{\linewidth}{llX}
  {[}RD01{]} & ESO-XXXXXX
             & CRIRES+ User Manual \\
\end{tabularx}

%%\bibliography{cr2repdoc}


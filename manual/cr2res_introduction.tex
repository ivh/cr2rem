\section{Introduction}
\label{sec:introduction}

\red{\textit{This template contains the basic layout of the User Manual and
    comments/instructions in ``red'' on how the different sections should
    be filled. They also indicate which parts are provided by ESO. These
    comments should be removed from the final document. If there are
    questions regarding the template, the instructions or the intended content
    of certain sections, please get in touch with your ESO contact
    person.}} 

\subsection{Scope}
\label{sec:scope}

\subsection{Acknowledgements}

\subsection{Stylistic conventions}
\label{sec:style}

Throughout this document the following stylistic conventions are used:

\begin{tabular}{lp{0.75\linewidth}}
\textbf{bold}     & in text sections for commands and other
                    user input which has to be typed as shown \\
\textit{italics}  & in the text and example sections for parts of the user
                    input which have to be replaced with real contents \\
\texttt{teletype} & in the text for FITS keywords, program names, file paths,
                    and terminal output, and as the general style for examples,
                    commands, code, etc \\
\end{tabular}

In example sections expected user input is indicated by a leading shell
prompt.

In the text \textbf{bold} and \textit{italics} may also be used to highlight
words.

\subsection{Notational Conventions}
\label{sec:notation}

Hierarchical FITS keyword names, appearing in the document, are given using the
dot--notation to improve readability. This means, that the prefix ``HIERARCH
ESO'' is left out, and the spaces separating the keyword name constituents in
the actual FITS header are replaced by a single dot.

\section{Related Documents}
\label{sec:doc-related}

\subsection{Applicable Documents}
\label{sec:doc-applicable}

\begin{tabularx}{\linewidth}{lllX}
  {[}AD01{]} & ESO-XXXXXX & 1.0
             & TBD \\
\end{tabularx}

\subsection{Reference Documents}
\label{sec:doc-reference}

\begin{tabularx}{\linewidth}{llX}
  {[}RD01{]} & ESO-XXXXXX
             & CR2RES User Manual \\
\end{tabularx}

%%\bibliography{iiinstrumentpdoc}


\section{The Instrument}
\label{sec:instrument}

\instrument{} is a major upgrade of the original CRyogenic Infra-Red Echelle
Spectrograph, henceforth \textit{oCRIRES}. The upgrade foremost introduces
\begin{itemize}
  \item A cross-diperser unit (CDU) with one grating per band (YJHKLM).
  \item New detectors, 3x 2048x2048.
  \item A spectropolarimeter (SPU) with beam-splitters in YJHK for both linear
        and circular polarization.
  \item New calibration sources (U-Ne-lamp, Fabry-Pérot, gas cell)
\end{itemize}

For details about the instrument's function and operation, we refer to the
\emph{User Manual} \cite{CIRESMAN} and only give a short overview over the
most important changes from the perspective of data reduction.

A look at the echellogram makes the most important aspects apparent:
\putgraph{\linewidth}{yband_etalon.png}{yetalon}{Example raw frame in J-band,
  with the regularly spaced lines from the Fabry-P\'erot etalon.}

\begin{itemize}
  \item There are up to 9 spectral orders (depending on band).
  \item The orders are spaced unevenly over the detectors.
  \item The orders are not perfectly horizontal.
  \item The projection of the slit is tilted, i.e. not aligned with detector
    columns. And the tilt changes with wavelength.
\end{itemize}

This in essence made necessary the development of a new DRS from scratch, around
robust algorithms for tracing the spectral orders, characterizing the slit tilt
and then making use of this information for optimally extracting the spectra
into 1D-arrays \cite{2021A&A...646A..32P}.

Within each order, wavelength increases from left to right over the three
detectors. Spectral orders with shorter (longer) wavelengths are located further
towards the top (bottom) of the pixel grid, i.e.~at higher (lower) Y pixel
values.

In order to increase the validity of daytime calibrations for observations from
the night, a metrology system is in place that iterates on the position of the
echellogram, both in main dispersion and cross-dispersion direction, thereby
improving repeatability.

Therefore, and to minimize the calibration overhead, \instrument\ is operated
with fixed \emph{standard settings}. These are chosen such that the full
wavelength range is covered in each band, as well as the detector gaps. In
general, the DRS does not rely on these settings, as long as it receives a set
of calibrations that match the data. However, only the standard settings are
supported for the time being and the DRS uses header keys like
\texttt{INS.WLEN.ID}\footnote{\texttt{INS.WLEN.ID} takes the form of a string
that starts with a letter to indicate the band (YJHKLM), followed by four digits
that represent the central wavelength, rounded.} for consistency checks, where
appropriate.


\begin{table}[htbp]
  \centering\begin{tabular}{cc}
  Y  & 2\\
  J  & 3\\
  H  & 4\\
  K  & 4\\
  L  & 7\\
  M  & 9\\
  \end{tabular}
  \caption{The number of standard settings per band.}
  \label{tab:nsettings}
\end{table}

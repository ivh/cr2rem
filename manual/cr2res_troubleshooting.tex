\section{Troubleshooting}
\label{sec:troubleshooting}

%\red{\textit{Hints on diagnosing problems and possible work-arounds or debug settings.}}


\subsection{Checking the TW}

In case the extracted spectra do not match the expectation from the raw frames,
the first thing is to check the traces in TW, i.e.~if the correct region gets
extracted.

This can be most easily done by plotting the traces on top of a
frame, either a raw frame, or an intermediate product like the nodding-combined
image with positive and negative target spectra from positions A and B,
respectively. Note that \texttt{cr2res\_obs\_nodding} also saves the traces that it derived for positions A and B.

Such a plot can be made with \texttt{cr2res\_util\_trace\_map} or one of the
scripts from App.~\ref{sec:scripts}.

Similarly, the slit-tilt in the TW should be checked if the resolution of
spectra is lower than it should be.


\subsection{Imperfect Order Traces}

The tracing of spectral orders can be less reliable when telluric absorption
very strong, like in some orders in L and M-bands. The parameters of
\texttt{cr2res\_util\_trace} allow to change the amount of smoothing in x and
y-direction, and the threshold for signal-detection. Varying these is the first
step to improve the traces.

Recipes that extract spectra into 1D have the parameter
\texttt{--extract\_height} which allows you to ignore the edge-detected upper
and lower polynomials and instead use a fixed extraction height, in pixels,
centered on the mid-line. This can be used as a work-around when the mid-line of
a trace is fine, but the edges are not.

If for some reason the traces cannot be derived from FLATs at all,
\texttt{cr2res\_util\_trace} can be used on the science frame itself. This works
well if the target has a reasonably strong continuum.

\subsection{Debug Settings}

The environment variable \texttt{ESOREX\_MSG\_LEVEL} controls how much logging
output is produced, and can take the values \texttt{off}, \texttt{error},
\texttt{warning}, \texttt{info} and \texttt{debug}. If it is set to
\texttt{debug}, the DRS not only produces extra logging output, but also dumps
some intermediate output into FITS files named as \texttt{debug\_*.fits}. These
are not documented, but can be self-explanatory.

Instead of the environment variable, the message level can also be set as
command option:
\begin{shell}
    %prompt esorex --msg-level=debug ... 
\end{shell}
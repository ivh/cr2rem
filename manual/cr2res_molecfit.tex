\section{Processing reduced CRIRES data with MOLECFIT}
\label{sec:molecfit}


\subsection*{Introduction}

Telluric lines - absorption from species in the Earth atmosphere - are strongly 
contaminating astronomical spectroscopic observation in the near-infrared, being
superimposed to the spectrum of the science target. The removal
of telluric lines from observations is therefore an important step in the reduction
and analysis of near-infrared spectra .

The removal of telluric lines is usually done either by observing a telluric standard star
close in time from the science target, or through the modelling of the telluric spectrum.
The latter method has the advantage of using no observing time. One tool of interest for 
ESO instruments which can model telluric spectrum and correct it in science spectra is {\tt MOLECFIT}
\footnote{{\tt MOLECFIT} can be downloaded at \url{https://www.eso.org/sci/software/pipelines/}} 
 \cite{smette-molecfit}.

The processing of CRIRES extracted spectra with {\tt MOLECFIT} is, at this point, neither
a part of the DRS nor supported by ESO. However, we provide here basic instructions and a custom script to convert
CRIRES reduced spectra into data which are ready to be processed with {\tt MOLECFIT}. This appendix
is not a tutorial, we assume that the user is familiar with {\tt MOLECFIT}. 

\subsection*{Instructions}

We assume that the user has an extracted spectrum from the DRS, hereafter called input spectrum. It can be a spectrum:
\begin{itemize}
  \item acquired in staring or nodding mode
  \item  if from nodding: either from A or B position, or combined
\end{itemize}

The python script \verb!cr2res_drs2molecfit! will change the format of the input spectrum so that it can be processed with
{\tt MOLECFIT}. The primary extension of the output spectrum will be identical to the input spectrum. However,
every subsequent FITS extention will contain a spectral order projected on a single detector in table format with
threw columns: WAVE, SPEC, and ERR containing respectively the wavelength in microns, the spectrum, and the error spectrum.

The script \verb!cr2res_drs2molecfit! also creates a corresponding 
\begin{itemize}
  \item a \verb!WAVE_INCLUDE.fits! file, indicating the wavelength
region to be included in the fit. By default, all the spectrum is included, the user can modify the file if they wish.
  \item a  \verb!ATM_PARAMETERS_EXT.fits! file
\end{itemize}

Make sure to edit the configuration file for the \verb!molecfit_model! recipe with:
\begin{itemize}
 \item \verb!COLUMN_LAMBDA=WAVE!
 \item \verb!COLUMN_FLUX=SPEC!
 \item \verb!COLUMN_DFLUX=ERR!
\end{itemize}
so that {\tt MOLECFIT} can map the columns from the SCIENCE.fits file to wavelength, flux, and error arrays.

\section{Data}
\label{sec:data}


%\red{\textit{Description of the different raw frames processed by the
%    pipeline, including data classification and association keywords (cf. example)}}

\subsection{Raw Data}

\begin{tabularx}{\linewidth}{|X|X|X|}
    \hline
    \multicolumn{1}{|l|}{\textbf{DPR.TYPE}} &
    \multicolumn{1}{l|}{\textbf{DPR.CATG}} &
    \multicolumn{1}{l|}{\textbf{Processed by}}\\
    \hline
    \texttt{DARK}                   & \texttt{CALIB} & \texttt{cr2res\_cal\_dark} \\
    \texttt{FLAT}                   & \texttt{CALIB} & \texttt{cr2res\_cal\_flat} \\
    \texttt{LAMP,METROLOGY}         & \texttt{CALIB} & None \\
    \texttt{WAVE,ABSORPTION\_SGC}   & \texttt{CALIB} & None, atm. \\
    \texttt{WAVE,ABSORPTION\_N2O}   & \texttt{CALIB} & None, atm. \\
    \texttt{WAVE,ABSORPTION-CELL}   & \texttt{CALIB} & None \\
    \texttt{WAVE,UNE}               & \texttt{CALIB} & \texttt{cr2res\_cal\_wave} \\
    \texttt{WAVE,FPET}              & \texttt{CALIB} & \texttt{cr2res\_cal\_wave} \\
    \texttt{WAVE,LAMP}              & \texttt{CALIB} & None. \\
    \texttt{FLAT,LAMP,DETCHECK}     & \texttt{CALIB} & \texttt{cr2res\_cal\_detlin} \\
    \texttt{DARK,DETCHECK}          & \texttt{CALIB} & \texttt{cr2res\_cal\_detlin} \\
    \hline
\end{tabularx}
\label{tab:raw-calibs}


The following table lists the files with \verb!DPR.CATG=SCIENCE!.

\begin{tabular}{|l|l|l|}
    \hline
    \textbf{DPR.TYPE} &
    \textbf{DPR.TECH} &
    \textbf{Processed by}  \\
    \hline
\texttt{OBJECT}   & \texttt{SPECTRUM,DIRECT,OTHER}   & \texttt{cr2res\_obs\_staring} \\
\texttt{OBJECT}   & \texttt{SPECTRUM,NODDING,JITTER} & \texttt{cr2res\_obs\_nodding} \\
\texttt{OBJECT}   & \texttt{SPECTRUM,NODDING,OTHER}  & \texttt{cr2res\_obs\_nodding} \\
\texttt{OBJECT}   & \texttt{SPECTRUM,NODDING,OTHER,ASTROMETRY}   & \texttt{cr2res\_obs\_nodding} \\
\texttt{OBJECT}   & \texttt{SPECTRUM,NODDING,OTHER,POLARIMETRY}  & \texttt{cr2res\_obs\_pol} \\
\texttt{OBJECT}   & \texttt{SPECTRUM,GENERIC}    & \texttt{cr2res\_obs\_2d} \\
\texttt{SKY   }   & \texttt{SPECTRUM,GENERIC}    & \texttt{cr2res\_obs\_2d} \\
\texttt{STD   }   & \texttt{SPECTRUM,NODDING,OTHER,POLARIMETRY}  & \texttt{cr2res\_obs\_pol} \\
\texttt{STD   }   & \texttt{SPECTRUM,NODDING,OTHER}  & \texttt{cr2res\_obs\_nodding} \\
    \hline
\end{tabular}
\label{tab:raw-science}


In addition to the recipes listed in the tables, any raw file can also be processed by \verb!cr2res_util_calib! to apply calibrations or combine frames.



%\subsection{Static Calibration Data}
%\label{sec:static-data}


\subsection{Data products}
\label{sec:data-prods}
\begin{tabularx}{\linewidth}{|X|X|X|X|}
    \hline
    \multicolumn{1}{|l|}{\textbf{PRO.TYPE}} &
    \multicolumn{1}{l|}{\textbf{PRO.CATG}} &
    \multicolumn{1}{l|}{\textbf{Remarks}} \\
    \hline
\texttt{BPM              } & \texttt{CAL\_DARK\_BPM}                & \\
\texttt{BPM              } & \texttt{CAL\_DETLIN\_BPM}              & \\
\texttt{BPM              } & \texttt{CAL\_FLAT\_BPM}                & \\
\texttt{BPM              } & \texttt{UTIL\_NORM\_BPM}               & \\
\texttt{CALIBRATED       } & \texttt{UTIL\_CALIB}                  & \\
\texttt{COMBINED         } & \texttt{OBS\_NODDING\_COMBINEDA}       & \\
\texttt{COMBINED         } & \texttt{OBS\_NODDING\_COMBINEDB}       & \\
\texttt{DETLIN\_COEFFS    } & \texttt{CAL\_DETLIN\_COEFFS}           & \\
\texttt{EXTRACT\_1D       } & \texttt{CAL\_FLAT\_EXTRACT\_1D}         & \\
\texttt{EXTRACT\_1D       } & \texttt{CAL\_WAVE\_EXTRACT\_1D}         & \\
\texttt{EXTRACT\_1D       } & \texttt{OBS\_NODDING\_EXTRACTA}        & \\
\texttt{EXTRACT\_1D       } & \texttt{OBS\_NODDING\_EXTRACTB}        & \\
\texttt{EXTRACT\_1D       } & \texttt{OBS\_NODDING\_EXTRACT\_COMB}    & \\
\texttt{EXTRACT\_1D       } & \texttt{UTIL\_EXTRACT\_1D}             & \\
\texttt{EXTRACT\_1D       } & \texttt{UTIL\_WAVE\_EXTRACT\_1D}        & \\
\texttt{LINES\_DIAGNOSTICS} & \texttt{CAL\_WAVE\_LINES\_DIAGNOSTICS}  & \\
\texttt{LINES\_DIAGNOSTICS} & \texttt{UTIL\_WAVE\_LINES\_DIAGNOSTICS} & \\
\texttt{MASTER\_DARK      } & \texttt{CAL\_DARK\_MASTER}             & \\
\texttt{MASTER\_FLAT      } & \texttt{CAL\_FLAT\_MASTER}             & \\
\texttt{MASTER\_FLAT      } & \texttt{UTIL\_MASTER\_FLAT}            & \\
\texttt{PRO.TYPE         } & \texttt{PRO.CATG}                    & \\
\texttt{SLIT\_CURV\_MAP    } & \texttt{UTIL\_SLIT\_CURV\_MAP}          & \\
\texttt{SLIT\_FUNC        } & \texttt{CAL\_FLAT\_SLIT\_FUNC}          & \\
\texttt{SLIT\_FUNC        } & \texttt{OBS\_NODDING\_SLITFUNCA}       & \\
\texttt{SLIT\_FUNC        } & \texttt{OBS\_NODDING\_SLITFUNCB}       & \\
\texttt{SLIT\_FUNC        } & \texttt{UTIL\_SLIT\_FUNC}              & \\
\texttt{SLIT\_MODEL       } & \texttt{CAL\_FLAT\_SLIT\_MODEL}         & \\
\texttt{SLIT\_MODEL       } & \texttt{OBS\_NODDING\_SLITMODELA}      & \\
\texttt{SLIT\_MODEL       } & \texttt{OBS\_NODDING\_SLITMODELB}      & \\
\texttt{SLIT\_MODEL       } & \texttt{UTIL\_SLIT\_MODEL}             & \\
\texttt{TW               } & \texttt{CAL\_FLAT\_TW}                 & \\
\texttt{TW               } & \texttt{CAL\_FLAT\_TW\_MERGED}          & \\
\texttt{TW               } & \texttt{CAL\_WAVE\_TW}                 & \\
\texttt{TW               } & \texttt{OBS\_NODDING\_TWA}             & \\
\texttt{TW               } & \texttt{OBS\_NODDING\_TWB}             & \\
\texttt{TW               } & \texttt{UTIL\_SLIT\_CURV\_TW}           & \\
\texttt{TW               } & \texttt{UTIL\_TRACE\_TW}               & \\
\texttt{TW               } & \texttt{UTIL\_WAVE\_TW}                & \\
\texttt{WAVE\_MAP         } & \texttt{CAL\_WAVE\_MAP}                & \\
\texttt{WAVE\_MAP         } & \texttt{UTIL\_WAVE\_MAP}               & \\
\texttt{CATALOG  } & \texttt{EMISSION\_LINES}               & from \texttt{cr2res\_util\_genlines} \\
\texttt{PHOTO\_FLUX  } & \texttt{PHOTO\_FLUX}               &  from \texttt{cr2res\_util\_genstd} \\
\hline
\end{tabularx}
\label{tab:cal-prods}

Note that the \verb'PRO.TYPE' defines the format and content of a file and \verb'PRO.CATG' name convention tells which recipe the file was made from.


The data products of \texttt{cr2res\_obs\_nodding} are named with suffixes as follows.
\begin{itemize}
    \item \verb!_OBS_NODDING_COMBINED[AB]! are the combined frames from A and B, one minus the other, i.e. one positive and one negative image, before extraction.
    \item \verb!_OBS_NODDING_TW[AB].fits! are the TW tables for extraction of A or B, respectively. These are made from the input TW, depending on the NODTHROW.
    \item \verb!_OBS_NODDING_EXTRACT[AB].fits! are the main data products, the extracted spectra from nodding positions A and B.
    \item \verb!_OBS_NODDING_EXTRACT_COMB.fits! is the last two combined, i.e. one re-sampled on the WL-scale of the other, and added together.
    \item \verb!_OBS_NODDING_SLITFUNC[AB].fits! is the (oversampled) slit-illumination function (for diagnostics only).
    \item \verb!_OBS_NODDING_SLITMODEL[AB].fits! is the 2D frame reconstructed from the slit-function and the spectrum (for diagnostics only).

\end{itemize}

Here, \verb![AB]! is shorthand for "either one of \verb!A! or \verb!B!", i.e.
the nodding positions.
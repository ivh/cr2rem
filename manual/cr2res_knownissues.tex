\section{Known Issues}
\label{sec:knownissues}

\subsection{Superresolution}

In good seeing conditions the AO system can produce a PSF that is sharp enough
to not evenly fill the slit width of the spectrograph. While this produces
spectra with a resolution higher than specified, it can also cause an offset in
the wavelength scale, because how the AO places the PSF center is not
reproducible enough. Wavelength shifts between reduced spectra from the nodding
positions A and B of around $1 pix$ have been encountered.
\footnote{This is not to be confused with the totally normal spectrum offset
due to slit tilt, see \ref{faq:tilt}}

This is not something that the DRS can correct for and will need to be mitigated
by allowing users to choose a "slit scanning" AO mode, in which the PSF center
is moved by small amounts between the slit edges during exposures.

Users can correct the wavelength offset between A and B spectra relatively
easily, by cross-correlating one to the other before combining them.

The following warning will be issued by the recipes \verb!cr2res_obs_nodding!
and \verb!cr2res_obs_staring! when it looks like this is the case, judging from
the illumination profile \emph{along} the slit. 
\begin{verbatim}
    Median FWMH of the PSF along the slit 
    is NN pix, i.e. below the slit width. This means the slit 
    is likely not evenly filled with light 
    in the spectral direction. This can result in a 
    wavelength offset between different postitions along the slit,
     and with respect to calibrations.
\end{verbatim}

\subsubsection{\ldots in polarimetry}

Although we are not aware of a data set that exhibits this, the wavelength-shift
due to superresolution can be expected to affect the two beams of the
polarimeter. The result after demodulation would be a non-zero polarization
signal in all spectral lines.

Such data would need more manual effort in data reduction, namely extracting the
beams one at a time and cross-correlating the spectra before demodulating.


\subsection{Calibration caveats in L and M band}

This is not a DRS-issue per se, but a current limitation of the instrument
calibration in general. The UNE and FPET cover only the bands YJHK, which means
that in L and M
\begin{itemize}
    \item the slit-tilt is not characterized, and spectrum extraction happens
    along a vertical slit.
    \item there is no wavelength calibration within the DRS. Instead, the
    wavelength scale from static calibrations (derived manually from telluric
    lines), is propagated to the data products.
\end{itemize}


\subsection{BPM in L and M}

The strong thermal emission present in the DARKs in L and M band makes the
detection of bad pixels more difficult than in the shorter bands. It makes sense
therefore to compare the derived BPM with one from e.g.~Y-band and vary the
parameter \verb!--bpm_kappa!.

Simply using the BPM from another band has also been found to work well,
cf.~\ref*{sec:bpmrefine}.


\section{Frequently Asked Questions}

\subsection{Spectra are shifted between nodding A and B}
\label{faq:tilt}

As mentioned throughout, the \instrument\ slit is not aligned with detector
columns, which means that spectra from different positions along the slit will
have different wavelength scales. The DRS takes care of this when assigning
wavelengths to spectral bins, therefore features in spectra from nodding A and B
should align when plotting against wavelength, but \emph{not} when plotting
against pixels.

\subsection{Overlapping spectra in IDP}

For the IDP data format (cf.~\ref{sec:idp}), the spectra from all
detector-orders in a setting are stored in single long arrays for wavelength,
ADU and errors, sorted by wavelength.

This naturally leads to large gaps in the spectrum, between detectors and
orders. In the Y-band, there is also some wavelength-overlap between orders and
this, in combination with the fact that spectra are not normalized, results in
large ``jumps'' in the spectrum in the overlap region, when plotting outright.

The solution is to use the additional data column, which gives the order number
for each spectral bin, to separate the overlap regions. Only after normalization
can the spectra be merged or spliced together, e.g.~using
\verb!cr2res_utiul_splice!.

\subsection{Barycentric correction}

The pipeline does currently not correct spectra for velocity shifts. Users
therefore need to carry out the barycentric velocity correction, if needed, with
an external tool.

It is however foreseen to use routines that were recently added to the
common pipeline library HDRL, in order to provide this information in 
data product headers.

\subsection{Timestamps}
For observations that comprise a time-sequence, e.g.~an exoplanet transit,
the timestamp of reduced data is relevant. The DRS does currently not 
calculate the (weighted) average timestamp of exposures that are combined. This ``midpoint'' in time will get added as a separate header key in the future.

In general, data products inherit their FITS headers from the \emph{first raw frame} in the input SOF, this includes \texttt{OBS.MJD}. Of course, the timestamps of the raw frames allow users to carry out the calculation manually.
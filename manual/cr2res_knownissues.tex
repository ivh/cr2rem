\section{Known Issues}
\label{sec:knownissues}

\subsection{Superresolution}
In good seeing conditions the AO system can produce a PSF that is sharp enough
to not evenly fill the slit width of the spectrograph. While this produces spectra with a resolution higher than specified, it can also cause an offset in the wavelength scale, because how the AO places the PSF center is not reproducible enough. Wavelength shifts between reduced spectra from the nodding positions A and B of around $1 pix$ have been encountered.

This is not something that the DRS can correct for and will need to be mitigated
by allowing users to choose a "slit scanning" AO mode, in which the PSF center
is moved by small amounts between the slit edges during exposures.

Users can correct the wavelength offset between A and B spectra relatively easily, by cross-correlating one to the other before combining them.

The following warning will be issued by the recipes \verb!cr2res_obs_nodding!
and \verb!cr2res_obs_staring! when it looks like this is the case, judging from
the illumination profile \emph{along} the slit. 
\begin{verbatim}
    Median FWMH of the PSF along the slit 
    is NN pix, i.e. below the slit width. This means the slit 
    is likely not evenly filled with light 
    in the spectral direction. This can result in a 
    wavelength offset between different postitions along the slit,
     and with respect to calibrations.
\end{verbatim}

\subsection{IDP / SDP}
The implementation of the IDP data format (cf.~\ref{sec:idp}) is available for the recipes \verb!cr2res_obs_nodding! and \verb!cr2res_obs_staring!, but not yet for \verb!cr2res_obs_2d! and \verb!cr2res_obs_pol!.

Futhermore, the following improvements are still outstanding

\begin{itemize}
    \item Flagging of spectral bins from detector edges and regions affected by ghosts.
    \item .
    
\end{itemize}

\subsection{Calibration caveats in L and M band}

This is not a DRS-issue per se, but a current limitation of the instrument
calibration in general. The UNE and FPET cover only the bands YJHK, which means
that in L and M
\begin{itemize}
    \item the slit-tilt is not characterized and spectrum extraction happens
    along a vertical slit.
    \item there is no wavelength calibration within the DRS. Instead the
    wavelength scale from static calibrations (derived manually from telluric
    lines), is propagated to the data products.
\end{itemize}


\subsection{BPM in L and M}

The strong thermal emission present in the DARKs in L and M band makes the
detection of bad pixels more difficult than in the shorter bands. It makes sense
therefore to compare the derived BPM with one from e.g.~Y-band and vary the
parameter \verb!--bpm_kappa!.

Simply using the BPM from another band has also been found to work well, cf.~\ref*{sec:bpmrefine}.

\subsection{Overestimated Errors}
\label{sec:erroroverest}
With the optimal extraction, the error spectrum is derived from the residuals
between extraction model and data (cf.~\ref{sec:errors}). It can therefore
happen, especially for spectra with low signal-to-noise, that the error spectrum
is too pessimistic, in the sense that dividing the spectrum by the error
spectrum gives a lower S/N (by up to a factor $\approx 3$) than is actually the
case (from photon statistics or measured from the spectrum RMS).

An improved error calculation has been implemented in version 1.3. In case the
issue persists, we recommend adjusting the relevant recipe parameters, namely
\begin{itemize}
    \item lowering the \emph{extraction height} as much as the data allows,
    \item increasing the \emph{swath width}, possibly up to the full detector
    size,
    \item setting \verb!--cosmics=TRUE! because outliers such as cosmic rays,
    while usually not affecting the spectrum itself, still contribute to the
    error estimate.
\end{itemize}


\section{Frequently Asked Questions}

\subsection{Barycentric correction}
The pipeline does currently not correct spectra for velocity shifts. Users therefore need to carry out the barycentric velocity correction, if needed, with an external tool.

It is however foreseen to use routines that were recently added to the
common pipeline library HDRL, in order to provide this information in 
data product headers.

\subsection{Timestamps}
For observations that comprise a time-sequence, e.g.~an exoplanet transit,
the timestamp of reduced data is relevant. The DRS does currently not 
calculate the (weighted) average timestamp of exposures that are combined. This ``midpoint'' in time will get added as a separate header key in the future.

In general, data products inherit their FITS headers from the \emph{first raw frame} in the input SOF, this includes \texttt{OBS.MJD}. Of course, the timestamps of the raw frames allow users to carry out the calculation manually.
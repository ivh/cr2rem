\section{Known Issues}
\label{sec:knownissues}

% \red{\textit{Know issues, problems and limitations related to the \textbf{current
%     release} of the pipeline (if possible with a work-around) go here.}}

\subsection{Calibration caveats in L and M band}

This is not a DRS-issue per se, but a current limitation of the instrument calibration in general. The UNE and FPET cover only the bands YJHK, which means that in L and M
\begin{itemize}
    \item the slit-tilt is not characterized and spectrum extraction happens along a vertical slit.
    \item there is no wavelength calibration within the DRS. Instead a static wavelength scale, present in the raw data headers, is propagated into the data products.
\end{itemize}


\subsection{BPM in L and M}

The strong thermal emission present in the DARKs in L and M band makes the detection of bad pixels more difficult than in the shorter bands. It makes sense therefore to compare the derived BPM with one from e.g.~Y-band and vary the parameter \verb!--bpm_kappa!.

Simply using the BPM from another band has also been found to work well.

%\subsection{Bugs}
%Issues that will be fixed in an upcoming release.

\section{Known Issues}
\label{sec:knownissues}

% \red{\textit{Know issues, problems and limitations related to the \textbf{current
%     release} of the pipeline (if possible with a work-around) go here.}}

\subsection{Calibration caveats in L and M band}

This is not a DRS-issue per se, but a current limitation of the instrument calibration in general. The UNE and FPET cover only the bands YJHK, which means that in L and M
\begin{itemize}
    \item the slit-tilt is not characterized and spectrum extraction happens along a vertical slit.
    \item there is no wavelength calibration within the DRS. Instead the wavelength scale from static calibrations (derived with \emph{molecfit}), is propagated to the data products.
\end{itemize}


\subsection{BPM in L and M}

The strong thermal emission present in the DARKs in L and M band makes the detection of bad pixels more difficult than in the shorter bands. It makes sense therefore to compare the derived BPM with one from e.g.~Y-band and vary the parameter \verb!--bpm_kappa!.

Simply using the BPM from another band has also been found to work well, cf.~\ref*{sec:bpmrefine}.

\subsection{Overestimated Errors in low S/N data}

For spectra with low signal-to-noise ($\lessapprox  10$) it is possible that the
normalization of the error spectrum is too pessimistic, i.e.~dividing the
spectrum by the error spectrum gives a lower S/N (by a factor $\approx 3$) than
is actually the case (from photon statistics or measured from the spectrum RMS).

Since the error is derived from the residual between extraction model and data, adjusting the relevant recipe parameters is expected to remedy the discrepancy, namely
\begin{itemize}
    \item lowering the \emph{extraction height} as much as the data allows, and/or
    \item increasing the \emph{swath width}, possilby up to the full detector size.
\end{itemize}

\section{Frequently Asked Questions}

\subsection{Barycentric correction}
The pipeline does currently not correct spectra for velocity shifts. Users therefore need to carry out the barycentric velocity correction, if needed, with an external tool.

It is however foreseen to use routines that were recently added to the
common pipeline library HDRL, in order to provide this information in 
data product headers.

\subsection{Timestamps}
For observations that comprise a time-sequence, e.g.~an exoplanet transit,
the timestamp of reduced data is relevant. The DRS does currently not 
calculate the (weighted) average timestamp of exposures that are combined. This ``midpoint'' in time will get added as a separate header key in the future.

In general, data products inherit their FITS headers from the \emph{first raw frame} in the input SOF, this includes \texttt{OBS.MJD}. Of course, the timestamps of the raw frames allow users to carry out the calculation manually.
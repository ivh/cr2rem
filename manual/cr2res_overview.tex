\section{Data Reduction Overview}
\label{sec:overview}

It makes sense to start from the end, that is to assume for a moment to
have valid master calibrations already available and look at the reduction of
the science frames first.

\subsection{Science reduction}
\label{sec:sci-reduc}

The recipes that reduce the science frames all are named as
\texttt{cr2res\_obs\_*}
all have in common that they receive as \textit{input} a set of raw science
frames, and the master calibrations, i.e. \textit{output} of calibration
recipes. They then first combine raw frames that belong together, for example
the frames from the same nodding position. Then the master calibrations are
applied, after which the appropriate algorithms for the data at hand are applied
to create the output data products.

The SOF to reduce a nodding sequence (three frames  each in positions A and B)
looks like:
\begin{verbatim}
$RAW/CRIRE.2021-08-16T23:22:50.334.fits OBS_NODDING_OTHER
$RAW/CRIRE.2021-08-16T23:23:54.728.fits OBS_NODDING_OTHER
$RAW/CRIRE.2021-08-16T23:24:58.960.fits OBS_NODDING_OTHER
$RAW/CRIRE.2021-08-16T23:26:18.394.fits OBS_NODDING_OTHER
$RAW/CRIRE.2021-08-16T23:27:22.616.fits OBS_NODDING_OTHER
$RAW/CRIRE.2021-08-16T23:28:26.956.fits OBS_NODDING_OTHER
$CALIB/K2192_masterflat.fits           CAL_MASTER_FLAT
$CALIB/K2192_tw.fits                   UTIL_WAVE_TW
$CALIB/cr2res_cal_dark_bpm.fits        CAL_DARK_BPM
$CALIB/cr2res_detlin_coeffs.fits       CAL_DETLIN_COEFFS
\end{verbatim}

It is then given to the recipe like this:
\begin{shell}
    # esorex cr2res_obs_nodding <SOF>
\end{shell}

For nodding observations of point sources, what happens next is that
position B is subtracted from position A and the so background-subtracted
spectra gets extracted into 1D arrays, using the supplied TW-table
for information on order trace, slit tilt and wavelength calibration.

The full list of observing recipes is:
\begin{itemize}
    \item \texttt{cr2res\_obs\_nodding} for nodding observations of
        point-sources. Also spectro-astrometry.
    \item \texttt{cr2res\_obs\_staring} for point-sources without nodding.
    \item \texttt{cr2res\_obs\_2d} for observations in which the spatial
        resolution along the slit should be conserved, i.e. no extraction.
    \item \texttt{cr2res\_obs\_pol} for observations with the spectro-polarimeter,
        including the demodulation into Stokes parameters. 
\end{itemize}


In general, data reduction is performed separately on each detector-order,
that is a single spectral order in a single detector.
Consequently, results are also stored this way in data products.


\subsection{Data Formats}
\label{sec:data-fmt-quick}

\subsubsection{Detectors and Extensions}
\label{sec:extns}
\begin{figure}[!tb]
  \begin{center}
    \includegraphics[width=0.99\linewidth]{fits_ext.pdf}
  \end{center}
  \caption{
    \label{fig:fits_ext}
    FITS extenstions in raw frames and data products.
    }
\end{figure}

Raw data comes as FITS files that contain no data, only headers in the
primary extension, and one image extension with the readout result from each
detector. The latter are named like \texttt{CHIPn.INT1} where \texttt{n} is
the detector number (1..3), in order from left to right with increasing
wavelength in each spectral order.

Data reduction products are either FITS tables or images, depending on whether
the product is an image/map or not. The separation into extensions of data from
the three detectors is kept, and most recipes treat them independently. For
derived maps or images that have errors, three more extension are present, named
like \texttt{CHIPnERR.INT1}.

Fig.~\ref{fig:fits_ext} illustrates this. The extensions are usually ordered,
however we recommend to no rely on the index when opening files with custom
scripts or tools. Instead, the FITS extension \emph{name} should be used.

\subsubsection{The TraceWave-Table}
\label{sec:tracewave}

\begin{figure}[!tb]
    \begin{center}
      \includegraphics[width=0.99\linewidth]{J_2_3_03_cr2res_util_trace_out.png}
    \end{center}
    \caption{
      \label{fig:flat_trace}
      An example of a raw flat-field frame in J/2/3. Overplotted in white is 
      the result of the order tracing, the polynomial fit to the mid-line
      (dashed) and edges (dotted) for each detector-order.
      }
  \end{figure}

The most important FITS table within the \instrument\ DRS is the
\emph{TraveWave}, abbreviated TW henceforth. It contains information about how
spectral orders are located in the detectors' pixel grid, how the projection of
the slit changes, and the wavelength solution.

Each row in the table corresponds to a single \emph{trace}. Each trace belongs
to a single spectral order; an order can have one or more traces, numbered
starting with 1 within each order. The \emph{order number} and \emph{trace
number} are two of the columns in the TW table, and together they uniquely
identify a trace, i.e.~a table row.

Each trace stores three polynomials: one for the mid-line, upper and lower edge,
respectively. Polynomials are stored as coefficients, lowest power first, and
evaluated for pixel columns starting with index 1.

Analogously, the wavelength solution is stored as polynomial coefficients that
represent the translation from pixel to wavelength space. Because the slit is
not vertical with respect to the pixel columns, the wavelength scales naturally
shifts within each spectral order. This is taken into account when the spectra
are extracted, so that the TW only needs to store the wavelength solution at the
mid-line of the trace.

How the orientation and shape of the slit changes along each trace is saved as
three "meta-polynomials" that are described in detail in TODO. Since this is handled automatically by the extraction methods, users normally should not need to get familiar with this aspect.

Several recipes take a TW as input and produce a new one, with some information
updated. Notably the wavecal recipe needs a TW as input to know how to extract
the spectra of the calibration lamps, and to get a first-guess of the
wavelength solution; it will then output a TW with the new wavelengths and
the other information propagated from the input.



\subsubsection{Pixels, Indices and Spectral Bins}
Pixel index starts at 1, not 0. Also and especially when evaluating polynomials like trace or wavecal.

Spectral bins correspond to detector columns at the mid-line of the extracted
region (constant height).

Orders are indexed from 1..9, not always starting at 1, depending on cut-off
orders. Index set by headers, conversion to order number $m$ by XXX...

\subsection{The Pipeline Recipes}
\label{sec:recipes-quick}

Our naming convention groups the recipes into
\begin{itemize}
    \item \textit{Observing recipes}, named like \texttt{cr2res\_obs\_*}, which
    receive raw science frames and pre-processed calibrations in order to
    produce the main science data products.
    \item \textit{Calibration recipes}, named like \texttt{cr2res\_cal\_*}. These are triggered online at Paranal and perform the steps necessary to produce the calibration products.
    \item The \textit{utility recipes}, named like \texttt{cr2res\_util\_*}, are a more diverse set of recipes. Some offer more fine-grained control over a single of the steps within a \texttt{cal\_} recipe; others exist to prepare static calibrations or perform additional tasks that are not covered by other recipes.
\end{itemize}

The following is a list of all recipes with a single-sentence description of their purpose. Details can be found in the \emph{man-pages} (Ch.~\ref{sec:manpages}) and Ch.~\ref{sec:recipes-reference}.



%%%
%%%%%%%%%%%%%% SCIENCE ABOVE  %%%%% CALIBS BELOW
%%%%

\subsection{Reducing Calibrations}
\label{sec:calib:reduc}

The following instrument properties need to be characterized:
\begin{itemize}
    \item Detector: non-linearity
    \item Detector: dark current
    \item Detector: bad pixel map
    \item Flat-fielding\footnote{By \textit{flat-field} we always mean the
              normalized pixel-to-pixel variations
              in sensitivity, not the response at larger scales.}
    \item The location and shape of the spectral orders on each detector.
    \item The orientation (tilt) of the slit within each order.
    \item Wavelength calibration.
\end{itemize}

The \emph{detector non-linearity} is derived from a long series of flat-field
frames. These are not taken on a frequent basis and are valid for a long time.
Users will almost always be able to use the provided master calibration and not
need to run this step.

The \emph{master dark} is ideally derived from dark exposures with the
same DIT, and as high NDIT as possible. For short exposures the dark current
itself is negligible, but the detector readout-mode leaves a residual pattern
that scales non-trivially with DIT, so that scaling dark by unmatched exposure
time is not recommended.

Darks exposures are not really "dark", but contain come background thermal emission,
especially in the longer bands. Therefore, Darks are considered specific to
each instrument setting.

The \emph{master flat} is derived from a halogen lamp exposure; it is specific
to each spectrograph setting, for obvious reasons. By \emph{flat-field} the DRS
means the \emph{normalized flat-field}, that is the pixel-to-pixel sensitivity
variations, alone. The wavelength-dependency within each spectral order is
saved as the \emph{blaze} to a separate data product.

\emph{Bad pixel masks} can come from several steps
\begin{itemize}
    \item darks, rejecting outliers like hot and dark pixels.
    \item flats, rejecting pixels outside a range of sensitivity.
    \item detlin, rejecting pixels with deviating linearity behaviour.
    \item edge pixels. Each detector has 4 colunms and rows at the edges that
        are tnot sensitive to light.
    \item inter-order, masking the pixels that are not part of a spectral order.
\end{itemize}

Therefore, several recipes produce BPMs and there are recipes that can merge
and separate them.

Except for the detector linearity, calibrations are valid only for a
single setting (band \& echelle) and there are a number of \textit{fixed
standard settings} defined that cover the spectral range in all the bands,
including the gaps between detectors. While DRS should in principle be able to
handle non-standard settings just fine, this is not supported for the time
being.


\begin{figure}[!tb]
    \begin{center}
        \includegraphics[width=0.99\linewidth]{calib_simple.pdf}
    \end{center}
    \caption{
        \label{fig:calibflow_simple}
        Flow diagram for calibration reduction with the \textit{calibration
        recipes}, assuming a populated CalibDB exists. Arrows and lines
        indicate the inputs to the recipes (blue), in the form of raw data
        (red), master calibrations from the CalibDB (green), or intermediate
        products from previous steps (white). Dotted lines indicate when a
        CalibDB or intermediate product can be chosen for later steps
        further to the right. All line connections are optional (open
        symbols), but generally recommended to be present. Also note that
        open diamonds mark Tracewave (TW) inputs, of which at least one
        needs to be present.
        \newline
        If a TW is provided to \texttt{cr2res\_cal\_flat}, it will use the
        information therein; otherwise it performs the tracing of the spectral
        orders to find their location. However, the slit tilt cannot be
        determined from flat-field frames which means that the extraction of the
        blaze function will be done assuming a vertical slit, yielding
        sub-optimal results. Therefore providing the TW from the CalibDB is
        recommended.
    }
\end{figure}


\begin{figure}[!tb]
    \begin{center}
        \includegraphics[width=0.99\linewidth]{calib_detailed.pdf}
    \end{center}
    \caption{
        \label{fig:calibflow_detailed}
        Flow diagram for calibration reduction with the \textit{utility
            recipes},
        without relying on any input from the CalibDB. In other words, this is
        how the products for the CalibDB (green) can be made from scratch.
        Steps
        are
        numbered but execution order can be changed whenever data flow allows.
        For
        notes on individual steps see main text.
    }
\end{figure}

Fig.~\ref{fig:calibflow_detailed} shows the calibration data flow cascade
step-by-step, using the \emph{util} recipes that only perform a single task
each:
\begin{enumerate}
    \item \texttt{cr2res\_util\_genlines} creates the catalogue files for each
          setting,
          based on the base catalogue and selection files that mark wavelength
          regions
          that are flagged as good for a certain setting.
    \item \texttt{cr2res\_cal\_detlin} takes flat-field exposures from
          different settings and with varying exposure times to measure the
          linearity
          behaviour of all pixels. Outliers get flagged as a bad-pixel-map
          (BPM).
    \item \texttt{cr2res\_cal\_dark} combines the input DARK frames into a
          master dark for
          each setting and combination of DIT and NDIT. Hot and dead pixels get
          saved as BPM.
    \item \texttt{cr2res\_util\_calib} combines the raw FLAT frames and applies
          the	       previous master calibs.
    \item \texttt{cr2res\_util\_trace} finds continuous regions on the
          detectors that
          have signal, and fits them with polynomials for the orders' mid-lines
          and edges.
          These polynomials we call a \emph{trace} and each order can have more
          than one,
          e.g. marking different heights along the slit.
    \item \texttt{cr2res\_util\_slit\_curv} uses the TraceWave-table (TW) from
          the previous
          step to measure the lines a FP-Etalon WAVE frame and determine how
          the
          slit tilt
          changes within each detector-order. The result is stored in an
          updated
          TW.
    \item \texttt{cr2res\_util\_extract} can now use this TW to extract the
          previously
          calibrated flat-field frame, resulting in the blaze spectrum and a
          model of
          the frame.
    \item This model contains all non-local features, meaning that
          \texttt{cr2res\_util\_normflat} can calculate the normalized
          flat-field
          by
          dividing the model by the the original frame. Outliers in sensitivity
          get
          flagged in another BPM.
    \item \texttt{cr2res\_util\_bpm\_merge} merges the BPMs from the different
          sources.
    \item \texttt{cr2res\_util\_calib} merges WAVE frames and applies the
          previous
          calibrations.
    \item \texttt{cr2res\_util\_extract} collapses the result from the last
          step into
          1D spectra.
    \item \texttt{cr2res\_util\_wave} uses these spectra and the catalogue of
          lines to
          derive a wavelength solution, i.e.~a polynomial that translates
          pixel-coordinates
          into wavelength (nm). Available methods include cross-correlation and
          line-fitting. This step can be repeated, for example to step-wise
          increase the
          polynomial degree.
    \item[13-15.] Analogous to steps 10-12 but for a FP-Etalon frame this time.
          The Etalon method of \texttt{cr2res\_util\_wave} will use the
          zero-point
          and dispersion from the incoming previous wavecal (in TW) and refine
          the
          higher degrees of the solution.
\end{enumerate}
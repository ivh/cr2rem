\section{Installation}
\label{installation}

This chapter gives generic instructions on how to obtain, build and install
the \instname\, pipeline. Even if this chapter is kept as up-to-date as much as 
possible, it may not be fully applicable to a particular release. This might
especially happen for patch releases. One is therefore advised to read the 
installation instructions delivered with the \instname\, pipeline distribution kit.
These release-specific instructions can be found in the file \ \texttt{README}
\ located in the top-level directory of the unpacked \instname\, pipeline source 
tree. The supported platforms are listed in Section \ref{PIP:platforms}. 
It is recommended reading through Section \ref{PIP:compile-howto} 
before starting the installation.

A bundled version of the \instname\, pipeline with all the required tools 
and an installer script is available from 
\ \texttt{http://www.eso.org/pipelines/}, \ for users 
who are not familiar with the installation of software packages.


\subsection{Supported platforms}
\label{PIP:platforms}

The utilisation of the GNU build tools should allow to build and install
the \instname\, pipeline on a variety of UNIX platforms, but it has only been
verified on the VLT target platforms:
\begin{itemize}
  \item Linux (glibc 2.1 or later),
  \item Sun Solaris 2.8 or later,
\end{itemize}
using the GNU C compiler (version 3.2 or newer).


\subsection{Building the IIINSTRUMENT pipeline}

This section shows how to obtain, build and install the \instname\, pipeline 
from the official source distribution.

\subsubsection{Requirements}
\label{PIP:requirements}

To compile and install the \instname\, pipeline one needs:
\begin{itemize}
  \item the GNU C compiler (version 3.2 or later),
  \item the GNU \ \texttt{gzip} \ data compression program,
  \item a version of the \ \texttt{tar} \ file-archiving program, and,
  \item the GNU \ \texttt{make} \ utility.
\end{itemize}

An installation of the Common Pipeline library (CPL) must also be
available on the system. Currently the CPL version 2.1.1 or newer is required.
The CPL distribution can be obtained from \cite{CPL}.

Please note that CPL itself depends on an existing qfits installation.
The qfits sources are available from the CPL download page or directly from
the qfits homepage at \ \texttt{http://www.eso.org/projects/aot/qfits}. \ In
conjunction with CPL 2.1.1 qfits 5.3.1 must be used.

In order to run the \instname\, pipeline recipes a front-end application is also
required. Currently there are two such applications available, a
command-line tool called \ \textit{EsoRex} \ and the Java based data file
organizer, \ \textit{Gasgano}, \ which provides an intuitive graphical user
interface (see Section \ref{LAUNCH}, page \pageref{LAUNCH}). At least one 
of them must be installed. The \ \textit{EsoRex} \ and
\ \textit{Gasgano} \ packages are available at
\ \texttt{http://www.eso.org/cpl/esorex.html} \ and
\ \texttt{http://www.eso.org/gasgano} \ respectively.

For installation instructions of any of the additional packages mentioned
before please refer to the documentation of these packages.


\subsubsection{Compiling and installing the IIINSTRUMENT pipeline}
\label{PIP:compile-howto}

The \instname\, pipeline distribution kit 1.0 contains:

\begin{tabbing}
xxxxxxx \=  xxxxxxxxxxxxxxxxxxxxxxxxxxxxxxxxxxxxxxxxxxxxx \= xxxxxxxxxxxxxxxxx \kill
\> \pipename-manual-1.0.pdf     \> The \instname\, pipeline manual      \\
\> install\_pipeline          \> Install script                 \\
\> cpl-\cplvers.tar.gz           \> CPL \cplvers                      \\
\> esorex-\esorexvers.tar.gz        \> esorex \esorexvers                   \\
\> gasgano-\gasganovers.tar.gz \> GASGANO \gasganovers for Linux        \\
\> \pipename-\pipelinevers.tar.gz         \> \instname\, \pipelinevers                    \\
\> \pipename-calib-\pipelinevers.tar.gz   \> \instname\, calibration files \pipelinevers  \\
\end{tabbing}

Here is a description of the installation procedure:

\begin{enumerate}
\item Change directory to where you want to retrieve the \instname\, pipeline 
recipes \pipelinevers package. It can be any directory of your choice but not: 
\begin{verbatim}
              $HOME/gasgano 
              $HOME/.esorex
\end{verbatim}

\item Download from the ESO ftp server, 
\ \texttt{http://www.eso.org/pipelines/},
\ the latest release of the \instname\, pipeline distribution.

\item Verify the checksum value of the tar file with the cksum command. 

\item Unpack using the following command:
\begin{verbatim}
           tar -xvf \pipename-kit-\pipelinevers.tar
\end{verbatim}

Note that the size of the installed software (including {\it Gasgano}) 
together with the static calibration data is about 27Mb. 

\item Install: after moving to the top installation directory,
\begin{verbatim}
           cd \pipename-kit-\pipelinevers
\end{verbatim}

it is possible to perform a simple installation using the available 
installer script ({\it recommended}):
\begin{verbatim}
           ./install_pipeline
\end{verbatim}

(beware: the execution may take a few minutes on Linux and several minutes 
on SunOS).

Note that this release still needs to link to the eclipse library.
At the end of the installation the user in addition to follow what reported
by the installation script, needs to source an file 
(\$HOME/..eclipse\_bash.rc or \$HOME/..eclipse\_bash.rc, depending from the
user shell) to set a few environment variables used by a few
low level eclipse library based modules.

By default the script will install the \instname\, recipes, \ {\it Gasgano}, 
\ {\it EsoRex}, all the necessary libraries, and the static calibration 
tables, into a directory tree rooted at \ {\tt \$HOME}. A different path 
may be specified as soon as the script is run.

The only exception to all this is the \ {\it Gasgano} \ tool, that will 
always be installed under the directory \ {\tt \$HOME/gasgano}. Note that 
the installer will move an existing \ {\tt \$HOME/gasgano} \ directory to 
\ {\tt \$HOME/gasgano.old} \ before the new \ {\it Gasgano} \ version is 
installed.

Important: the installation script would ensure that any existing 
\ {\it Gasgano} \  and \ {\it EsoRex} \ setup would be inherited 
into the newly installed configuration files (avoiding in this way 
any conflict with other installed instrument pipelines).
\end{enumerate}

Alternatively, it is possible to perform a manual installation 
({\it experienced users only}): the \ {\tt README} \ file located 
in the top installation directory contains more detailed information 
about a step-by-step installation. 

%%% Local Variables: 
%%% mode: latex
%%% TeX-master: t
%%% End: 

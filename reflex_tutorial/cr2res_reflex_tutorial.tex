\documentclass[pdftex,a4paper,twoside,11pt]{article}
\NeedsTeXFormat{LaTeX2e}

%% Include layout, additional commands and abbreviations
%%
%% Layout definitions
%%

\usepackage[english]{babel}
\usepackage{lastpage}
\usepackage{multirow}
\usepackage{longtable}
\usepackage{fancyhdr}
\usepackage{graphicx}
\usepackage{times}
\usepackage{verbatim}
%\usepackage[draft]{dmd-doc}
\usepackage{dmd-doc}

\renewcommand{\encodingdefault}{T1}

%%% Local Variables: 
%%% mode: latex
%%% TeX-master: t
%%% End: 

%% $Id: shortcut.tex,v 1.10 2013-03-27 10:13:54 cgarcia Exp $
%% Abbreviations and extra command definitions
%%

%%
%% Release information
%%

\newcommand{\releasedate}{March 2021}
\newcommand{\releaseday}{2021-03-03}
\newcommand{\releasetime}{03/Mar/2021}
\newcommand{\pipelinevers}{0.9.0}
\newcommand{\pipelinemanualvers}{0.9.0}

\newcommand{\cplvers}{7.0}
\newcommand{\cfitsiovers}{3360}
\newcommand{\esorexvers}{3.12.3}
\newcommand{\gasganovers}{2.4.8}
\newcommand{\reflexvers}{2.9.0}

%%
%% Abbreviations
%%

\newcommand{\eg}{{e.g.~}}
\newcommand{\ie}{{i.e.~}}

\newcommand{\degr}{\hbox{$^\circ$}}
\newcommand{\arcmin}{\hbox{$^\prime$}}
\newcommand{\arcsec}{\hbox{$^{\prime\prime}$}}

\newcommand{\CPL}{\textit{Common Pipeline Library}}

\newcommand{\tcen}{\raisebox{-7pt}[7pt][0pt]}

\newcommand{\instname}{CR2RES}
\newcommand{\pipename}{cr2res}
\newcommand{\wkfinstname}{CR2RES}

%%
%% Additional commands
%%

\newcommand{\cm}[1]{\marginpar{\scriptsize #1}}

\newcommand{\tbspa}{\rule[1ex]{0pt}{1.1ex}}
\newcommand{\tbspb}{\rule[-1.0ex]{0pt}{1.0ex}}

\newcommand{\http}[1]{\href{http://#1}{\textit{http://#1}}}
\newcommand{\ftp}[1]{\href{ftp://#1}{\textit{ftp://#1}}}
\newcommand{\mail}[1]{\href{mailto:#1}{\textit{#1}}}


\newcommand{\putgraph}[4]{
    \begin{figure}[ht]
        \begin{center}
    \includegraphics[width=#1truecm]{#2}
    \end{center}
    \caption{\it #4.}
    \label{fig:#3}
    \end{figure}
}

\newcommand{\putgraphhere}[4]{
    \begin{figure}[h!]
        \begin{center}
    \includegraphics[width=#1truecm]{#2}
    \end{center}
    \caption{\it #4.}
    \label{fig:#3}
    \end{figure}
}




\dmdProgram{GEN}
\dmdProject{Science Operation Software Department}
\dmdTitle{\instrument\ Reflex Tutorial}
\dmdDocId{ESO-413897}
\dmdDocVersion{1.4.0}
\dmdDocType{Manual (MAN)}
\dmdDocDate{2022-11-30}

\dmdPreparedBy{\instrument\ Pipeline Team}
\dmdValidatedBy{}
\dmdApprovedBy{}

%% Additional commands and abbreviations
\graphicspath{{./figures/}{./pipedoc/figures/}}

%% Remember to update shortcut.tex
\def\reldate{\releasedate}
\def\pipeno{\pipelinevers}
\def\kitno{\pipeno}
\def\docno{\pipelinemanualvers}
\def\qfitsno{\qfitsvers}
\def\cplno{\cplvers}
\def\esorexno{\esorexvers}
\def\reflexno{\reflexvers}
\def\gasganono{\gasganovers}
\def\jdkno{\jdkvers}
\def\dmdissue{\docno}

\setlongtables
\makeindex
\bibliographystyle{plain} 

\begin{document}
\pagenumbering{arabic}
\dmdmaketitle
\emptypage{This page was intentionally left blank}

\section*{Authors}
\begin{tabularx}{\linewidth}{|p{0.25\linewidth}|X|}
  \hline
  \multicolumn{1}{|l|}{\textbf{Name}}\tbspa &
  \multicolumn{1}{l|}{\textbf{Affiliation}} \tbspb \\
  \hline
  \tbspa
  Thomas Marquart & Uppsala University 
  \tbspb\\
  \tbspa
    Yves Jung & ESO   \tbspb\\
    \tbspa
    Valentin Ivanov & ESO   \tbspb\\
  \hline
\end{tabularx}

\emptypage{This page was intentionally left blank}

\section*{Change Record from previous Version}
\begin{tabularx}{\linewidth}{|p{0.25\linewidth}|X|}
  \hline
  \multicolumn{1}{|l|}{\textbf{Affected Section(s)}}\tbspa &
  \multicolumn{1}{l|}{\textbf{Changes/Reason/Remarks}}\tbspb \\
  \hline
  \tbspa
  All                      & First Version  \tbspb\\
  1.2.2                    & Additional workflows ans demo data \\
  \hline
\end{tabularx}
\emptypage{This page was intentionally left blank}

\tableofcontents
\cleardoublepage

\section{Introduction And Scope}

{\tt Reflex}\footnote{User support for this software is available by sending 
enquiries to \mail{usd-help@eso.org}} is the ESO Recipe Flexible Execution 
Workbench, an environment to run ESO VLT pipelines which employs a workflow 
engine (Kepler\footnote{\http{kepler-project.org}}) to provide a real-time
visual representation of a data reduction cascade, called a workflow,
which can be easily understood by most astronomers.  This document is
a tutorial designed to enable the user to employ the \instname\, workflow to
reduce his/her data in a user-friendly way, concentrating on
high-level issues such as data reduction quality and signal-to-noise 
(S/N) optimisation.

A workflow accepts science and calibration data, as delivered to PIs
in the form of PI-Packs (until October 2011) or downloaded from the
archive using the CalSelector
tool\footnote{\http{www.eso.org/sci/archive/calselectorInfo.html}} and
organises them into DataSets, where each DataSet contains one
science object observation (possibly consisting of several science
files) and all associated raw and static calibrations required for a
successful data reduction. The data organisation process is fully
automatic, which is a major time-saving feature provided by the
software. The DataSets selected by the user for reduction are fed
through the workflow which executes the relevant pipeline recipes (or
stages) in the correct order.
%, providing optional user interactivity at
%key data reduction points with the aim of enabling the iteration of
%certain recipes in order to obtain better results. 
Full control of the
various recipe parameters is available within the workflow, and the
workflow deals automatically with optional recipe inputs via built-in
conditional branches. Additionally, the workflow stores the reduced
final data products in a logically organised directory structure and
employing user-configurable file names.
 %This file is in the pipedoc directory

The workflow follows the "simple way" of reducing calibrations from the
\instrument\ Pipeline User Manual. That is it makes use of the TW-table from the
CalibDB and runs a single recipe for each of: darks, flats, wavecal and science
reduction.

\section{Software Installation}

The software pre-requisites for {\tt Reflex \reflexvers} may be found at:\newline
  \http{www.eso.org/sci/software/pipelines/reflex\_workflows}

To install the {\tt Reflex \reflexvers} software and demo data, 
please follow these instructions:
\begin{enumerate}
  \item From any directory, download the installation script:
        {\small
        \begin{verbatim}
        wget ftp://ftp.eso.org/pub/dfs/reflex/install_reflex
        \end{verbatim}
        }

  \item Make the installation script executable:
        {\small
        \begin{verbatim}
        chmod u+x install_reflex
        \end{verbatim}
        }

  \item Execute the installation script:
        {\small
        \begin{verbatim}
        ./install_reflex
        \end{verbatim}
        }
        and the script will ask you to specify three directories: the download
        directory {\tt \verb|<|download\_dir\verb|>|}, the software
        installation directory {\tt \verb|<|install\_dir\verb|>|}, 
        and the directory to be used to store the demo data 
        {\tt \verb|<|data\_dir\verb|>|}.
        If you do not specify these directories, then the installation script 
        will create them in the current directory with default names.

  \item You will be given a choice of pipelines (with the corresponding 
        workflows) to install. Please specify the numbers for the pipelines 
        you require, separated by a space, or type ``A'' for all pipelines.

  \item To start {\tt Reflex}, issue the command:
        {\small
        \begin{verbatim}
        <install_dir>/bin/reflex
        \end{verbatim}
        }
        It may also be desirable to set up an alias command for starting the 
        {\tt Reflex} software, using the shell command {\tt alias}. 
        Alternatively, the {\tt PATH} variable can be updated to contain the
        {\tt \verb|<|install\_dir\verb|>|/bin} directory.
\end{enumerate}
 %This file is in the pipedoc directory

\section{Quick Start: Reducing The Demo Data \label{sec:quick_start}}

The content of demo data sets are described in Table \ref{tab:demodata}.
Digits in the brackets give the number of raw frames of the given type.
Some calibration files are shared among multiple datasets.


\begin{tabular}{|l|l|l|l|l|l|}
\hline
Mode & OBS.TARG.NAME & INS.WLEN.ID & Science frames & Calib frames \\
2d & gam Gru & K2192  & object (2), sky (2) & flat (1), dark (2), UNe lamp (1), FPET (1)\\
stare & HD 201585 & K2166  & object (1) & flat (1), dark (3), UNe lamp (1), FPET (1)\\
nodding & CD-40 9712 & K2192  & object (2) & flat (1), dark (3), UNe lamp (1), FPET (1)\\
nodding & HD 222925 & K2192  & object (2) & flat (1), dark (2), UNe lamp (1), FPET (1)\\
nodding & HD 222925 & Y1028  & object (2) & flat (1), dark (3), UNe lamp (1), FPET (1)\\
nodding & alf Eri & L3262  & object (2) & flat (1), dark (1)\\
polarim. & eps CMa & K2192  & object (8) & flat (1), dark (3), UNe lamp (1), FPET (1) \\
\hline
\end{tabular}
\label{tab:demodata}


\putgraph{0.98\linewidth}{workflow_nodding.png}{pipe_wkf_layout}
    {Screenshot of the nodding workflow}

\putgraph{0.68\linewidth}{workflow_nodding_sub.jpg}{wkf_nodd_sub}
    {Screenshot of the expansion of the nodding-recipe box}

Fig.~\ref{fig:pipe_wkf_layout} and ~\ref{fig:wkf_nodd_sub} show the nodding data workflow
and the
sub-workflow that actually calls the nodding recipe, respectively.
The counterpart workflows for the other observing modes are idential,
except for the names of the corresponding science recipe.

%%% more screenshots, not used for now.
%\putgraph{0.98\linewidth}{workflow_2d.jpg}{wkf_2d}
%    {Screenshot of the 2D-workflow}
%\putgraph{0.98\linewidth}{workflow_stare.jpg}{wkf_stare}
%    {Screenshot of the staring workflow}
%\putgraph{0.98\linewidth}{workflow_pol.jpg}{wkf_pol}
%    {Screenshot of the polarimetry workflow}


For the user who is keen on starting reductions without being
distracted by detailed documentation, we describe the steps to be
performed to reduce the science data provided in the \instname\, demo
data set 
supplied with the \reflex\ {\tt \reflexvers} release. By following these
steps, the user should have enough information to perform a reduction
of his/her own data without any further reading:

\begin{enumerate}
  \item First, type:
        \begin{verbatim}
        esoreflex -l
        \end{verbatim}

    If the \reflex\ executable is not in your path, then you have
      to provide the command with the executable full path {\tt
        <install\_dir>/bin/esoreflex -l }. For convenience, we will
      drop the reference to {\tt <install\_dir>}. A list with the
      available \reflex\ workflows will appear, showing the workflow
      names and their full path.

   \item Open the \wkfname\ by typing:

\bigskip

      {\tt \ \ \ \ \ \ \ \ esoreflex} \wkfn {\tt \&}

\bigskip

      Alternatively, you can type only the command \reflex\, the empty
      canvas will appear (Figure~\ref{fig:reflex_empty}) and you can
      select the workflow to open by clicking on {\tt File -> Open
        File}. Note that the loaded workflow will appear in a new
      window. The \wkfname\ workflow is shown in
      Figure~\ref{fig:pipe_wkf_layout}. 

  \item To aid in the visual tracking of the reduction cascade, it is advisable
  to use component (or actor) highlighting. Click on {\tt Tools -> Animate at
  Runtime}, enter the number of milliseconds representing the animation
  interval (100\,ms is recommended), and click \fbox{\tt OK}.

\item Change directories set-up. Under ``Setup Directories'' in the
  workflow canvas there are seven parameters that specify important
  directories (green dots).

  By default, the {\tt ROOT\_DATA\_DIR}, which specifies the working
  directory within which the other directories are organised. is set
  to your {\tt \$HOME/reflex\_data} directory. All the temporary and
  final products of the reduction will be organized under
  sub-directories of {\tt ROOT\_DATA\_DIR}, therefore make sure this
  parameter points to a location where there is enough disk space. To
  change {\tt ROOT\_DATA\_DIR}, double click on it and a pop-up window
  will appear allowing you to modify the directory string, which you
  may either edit directly, or use the \fbox{\tt Browse} button to
  select the directory from a file browser.  When you have finished,
  click \fbox{\tt OK} to save your changes.

    Changing the value of {\tt RAW\_DATA\_DIR} is the only necessary
    modification if you want to process data other than the demo data

  \item Click the 
  \includegraphics[width=0.5cm,height=0.5cm]{reflex_run_button.png}
  button to start the workflow

\item The workflow will highlight the {\tt Data Organiser} actor which
  recursively scans the raw data directory (specified by the parameter
  {\tt RAW\_DATA\_DIR} under ``Setup Directories'' in the workflow
  canvas) and constructs the datasets. Note that the raw and static
  calibration data must be present either in {\tt RAW\_DATA\_DIR} or
  in {\tt CALIB\_DATA\_DIR}, otherwise datasets may be incomplete and
  cannot be processed. However, if the same reference file was
  downloaded twice to different places this creates a problem as
  \reflex\ cannot decide which one to use.

\item The {\tt Data Set Chooser} actor will be highlighted next and
  will display a ``Select Datasets'' window that lists the datasets
  along with the values of a selection of useful header
  keywords\footnote{The keywords listed can be changed by double
    clicking on the {\tt DataOrganiser} Actor and editing the list of
    keywords in the second line of the pop-up window. Alternatively,
    instead of double-clicking, you can press the right mouse button
    on the {\tt DataOrganiser} Actor and select {\tt Configure Actor} to
    visualize the pop-up window.}.  The first column consists of a set
  of tick boxes which allow the user to select the datasets to be
  processed. By default all complete datasets which have not yet been
  reduced will be selected.

  \item Click the \fbox{\tt Continue} button and watch the progress of
    the workflow by following the red highlighting of the actors. A
    window will show which dataset is currently being processed.

\ifdefined \instquickitemlist 
  \input{\instquickitemlist}
\fi

\item Once the reduction of all datasets has finished, a pop-up window
  called {\sl Product Explorer} will appear, showing the datasets
  which have been reduced together with the list of final products.
  This actor allows the user to inspect the final data products, as
  well as to search and inspect the input data used to create any of
  the products of the workflow.

\item After the workflow has finished, all the products from all the
  datasets can be found in a directory under {\tt END\_PRODUCTS\_DIR}
  named after the workflow start timestamp. Further subdirectories
  will be found with the name of each dataset.

\end{enumerate}

Well done! You have successfully completed the quick start section and
you should be able to use this knowledge to reduce your own
data. However, there are many interesting features of {\tt Reflex} and
the \instname\, workflow that merit a look at the rest of this tutorial.

\putgraph{8truecm}{reflex_empty_canvas.png}{reflex_empty}{The empty {\tt
    Reflex} canvas}



%%%% INPUT EXTERNAL FILES

%\section{Quick Start: Reducing The Demo Data \label{sec:quick_start}}

For the user who is keen on starting reductions without being
distracted by detailed documentation, we describe the steps to be
performed to reduce the science data provided in the \instname\, demo
data set 
supplied with the {\tt Reflex~\reflexvers} release. By following these
steps, the user should have enough information to attempt a reduction
of his/her own data without any further reading:

\putgraph{8}{reflex_empty_canvas.png}{reflex_empty}{The empty {\tt
    Reflex} canvas}
\putgraph{14}{reflex_pipewkf_layout.png}{pipe_wkf_layout}{\instname\,
  workflow general layout}

%Esto estaba con \begin{figure}[htb]
\putgraph{6}{reflex_select_data_sets.png}{reflex_select_data_sets}{The
``Select Datasets'' pop-up window}

\begin{enumerate}
  \item Start the {\tt Reflex} application:
        \begin{verbatim}
        reflex &
        \end{verbatim}
        The empty {\tt Reflex} canvas as shown in Figure~\ref{fig:reflex_empty}
        will appear.

  \item Now open the \instname\, workflow by clicking on {\tt File -> Open
    File}, selecting first {\tt \pipename-\pipelinevers} and then the file {\tt
    \wkffilename} in the file browser.  You will be presented with the
    workflow canvas shown in Figure~\ref{fig:pipe_wkf_layout}. Note that
    the workflow will appear as a canvas in a new window.

  \item To aid in the visual tracking of the reduction cascade, it is advisable
  to use component (or actor) highlighting. Click on {\tt Tools -> Animate at
  Runtime}, enter the number of milliseconds representing the animation
  interval (100\,ms is recommended), and click \fbox{\tt OK}.

  \item Under ``Setup Directories'' in the workflow canvas there are
    seven parameters that specify important directories (green
    dots). Setting the value of {\tt ROOT\_DATA\_DIR} is the only
    necessary modification if you want to process data other than the
    demo data\footnote{If you used the install script {\tt
        install\_reflex}, then the value of the parameter {\tt
        ROOT\_DATA\_DIR} will already be set correctly to the
      directory where the demo data was downloaded.}, since
    the value of this parameter specifies the working directory within
    which the other directories are organised. Double-click on the
    parameter {\tt ROOT\_DATA\_DIR} and a pop-up window will appear
    allowing you to modify the directory string, which you may either
    edit directly, or use the \fbox{\tt Browse} button to select the
    directory from a file browser.  When you have finished, click
    \fbox{\tt OK} to save your changes.

  \item Click the 
  \includegraphics[width=0.5cm,height=0.5cm]{reflex_run_button.png}
  button to start the workflow

  \item The workflow will highlight the {\tt Data Organiser} actor
    which has recursively scanned the raw data directory (specified by the
    parameter {\tt RAWDATA\_DIR} under ``Setup Directories'' in the
    workflow canvas) and constructs the DataSets. Note that the
    calibration and reference data must be present either in {\tt
      RAWDATA\_DIR} or in {\tt CALIB\_DATA\_DIR}, otherwise DataSets
    may be incomplete and cannot be processed. However, if the same
    reference file was downloaded twice in different places this
    creates a problem as {\tt Reflex} cannot decide which one to use.

  \item The {\tt Data Set Chooser} actor will be highlighted next and
    will display a ``Select Datasets'' window (see
    Figure~\ref{fig:reflex_select_data_sets}) that lists the DataSets
    along with the values of a selection of useful header
    keywords\footnote{The keywords listed can be changed by
      right-clicking on the {\tt DataOrganiser} Actor, selecting {\tt
        Configure Actor}, and then changing the list of keywords in
      the second line of the pop-up window. Make sure that the {\tt
        Lazy Mode} is not active and then click on \fbox{\tt Commit}
      to save the change.}  namely the object name, grism and filter
    name, information about the slit/mask (INS.MASK.ID for MXU,
    INS.MOS.CHECKSUM for MOS, and INS.SLIT.NAME for LSS), binning, and
    detector name.  The first column consists of a set of tick boxes
    which allow the user to select the DataSets to be processed, and
    by default all DataSets are selected.

  \item Click the \fbox{\tt Continue} button and watch the progress of
    the workflow by following the red highlighting of the actors. A
    window will show which DataSet is currently being processed.

\end{enumerate}

Well done! You have successfully completed the quick start section and
you should be able to use this knowledge to reduce your own
data. However, there are many interesting features of {\tt Reflex} and
the \instname\, workflow that merit a look at the rest of this tutorial.
 %This file is in the pipedoc directory
\section{The \instname\, Workflow \label{sec:wkf_general_desc}}

The \instname\,  workflow canvas is organised into a number of areas.
From top-left to top-right you
will find general workflow instructions, directory parameters, and
global parameters.  In the middle row you will find five boxes
describing the workflow general processing steps in order from left to
right, and below this the workflow actors themselves are organised
following the workflow general steps. 

\subsection{Workflow Canvas Parameters \label{sec:wkf_canvpar}}

The workflow canvas displays a number of parameters that may be set by
the user. Under ``Setup
Directories'' the user is only required to set the {\tt
  ROOT\_DATA\_DIR} to the working directory for the DataSet(s) to be
reduced, which, by default, is set to the directory containing the
demo data. Raw data should be stored in a subdirectory of {\tt
  ROOT\_DATA\_DIR}, defined by the parameter {\tt RAWDATA\_DIR}, which
is recursively scanned by the {\tt Data Organiser} actor for input raw
data. The directory {\tt CALIB\_DATA\_DIR}, which is within the
pipeline installation directory, is also scanned by the {\tt Data
  Organiser} actor to find any static calibrations that may be missing
in your DataSet(s).  If required, the user may edit the directories
{\tt BOOKKEEPING\_DIR}, {\tt LOGS\_DIR}, {\tt TMP\_PRODUCTS\_DIR}, and
{\tt END\_PRODUCTS\_DIR}, which correspond to the directories where
book-keeping files, logs, temporary products and end products are
stored, respectively (see the Reflex User Manual for further details;
\cite{REFLEXMAN}).

Under the ``Global Parameters'' area of the workflow canvas, the user
may set the {\tt FITS\_VIEWER} parameter to the command used for
running his/her favourite application for inspecting FITS
files. Currently this is set by default to {\tt fv}, but other
applications, such as {\tt ds9}, {\tt skycat} and {\tt gaia} for
example, may be useful for inspecting image data.

By default the {\tt EraseDirs} parameter is set to {\tt false}, which
means that no directories are cleaned before executing the workflow,
and the recipe actors will work in Lazy mode (see
Section~\ref{sec:lazy_mode}), reusing the previous pipeline recipe outputs
where input files and parameters are the same as for the previous
execution, which saves considerable processing time. Sometimes it is
desirable to set the {\tt EraseDirs} parameter to {\tt true}, which
forces the workflow to recursively delete the contents of the
directories specified by {\tt BOOKKEEPING\_DIR}, {\tt LOGS\_DIR}, and
{\tt TMP\_PRODUCTS\_DIR}. This is useful for keeping disk space usage
to a minimum and will force the workflow to fully rereduce the data
each time the workflow is run.


 %This file is in the pipedoc directory
\section{About The Reflex Canvas \label{sec:about_canvas}}

\subsection{Saving And Loading Workflows}

In the course of your data reductions, it is likely that you will
customise the workflow for various data sets, even if this simply consists of 
editing the {\tt ROOT\_DATA\_DIR} to a different value for each data set. 
Whenever you modify a workflow in any way, you have the option of saving the 
modified version to an {\tt XML} file using {\tt File -> Export As} 
(which will also open a new workflow canvas corresponding to the saved file). 
The saved workflow may be opened in subsequent
{\tt Reflex} sessions using {\tt File -> Open}. Saving the workflow in the
default format (.kar) is only advised if you do not plan to use the workflow
in other computer.

\subsection{Buttons}  
         
At the top of the {\tt Reflex} canvas are a set of buttons which have the following useful functions:
\begin{itemize}
\item{\includegraphics[width=0.5cm,height=0.5cm]{reflex_zoom_in_button.png} - Zoom in.}
\item{\includegraphics[width=0.5cm,height=0.5cm]{reflex_zoom_reset_button.png} - Reset the zoom to 100\%.}
\item{\includegraphics[width=0.5cm,height=0.5cm]{reflex_zoom_to_fit_button.png} - Zoom the workflow to fit the current window size (Recommended).}
\item{\includegraphics[width=0.5cm,height=0.5cm]{reflex_zoom_out_button.png} - Zoom out.}
\item{\includegraphics[width=0.5cm,height=0.5cm]{reflex_run_button.png} - Run (or resume) the workflow.}
\item{\includegraphics[width=0.5cm,height=0.5cm]{reflex_pause_button.png} - Pause the workflow execution.}
\item{\includegraphics[width=0.5cm,height=0.5cm]{reflex_stop_button.png} - Stop the workflow execution.}
\end{itemize}
The remainder of the buttons (not shown here) are not relevant to the 
workflow execution.

\subsection{Workflow States}

A workflow may only be in one of three states: executing, paused, or stopped.
 These states are indicated by the yellow highlighting of the
\includegraphics[width=0.5cm,height=0.5cm]{reflex_run_button.png},
\includegraphics[width=0.5cm,height=0.5cm]{reflex_pause_button.png},
and \includegraphics[width=0.5cm,height=0.5cm]{reflex_stop_button.png}
buttons, respectively. A workflow is executed by clicking the
\includegraphics[width=0.5cm,height=0.5cm]{reflex_run_button.png} button. 
Subsequently the workflow and any running
pipeline recipe may be stopped immediately by clicking the
\includegraphics[width=0.5cm,height=0.5cm]{reflex_stop_button.png}
button, or the workflow may be paused by clicking the 
\includegraphics[width=0.5cm,height=0.5cm]{reflex_pause_button.png} button which
will allow the current actor/recipe to finish execution before the workflow is 
actually paused. Note that after clicking the
\includegraphics[width=0.5cm,height=0.5cm]{reflex_pause_button.png} button, 
it is possible that more than one actor is executed, since this behaviour 
depends on the workflow scheduling. For instance, if there are two actors in 
parallel, and you pause the workflow while one is being executed, then both
of them will be executed before the workflow is actually paused. After pausing,
the workflow may be resumed by clicking the
\includegraphics[width=0.5cm,height=0.5cm]{reflex_run_button.png} button again.

\subsection{The Runtime Window}

You may find the runtime window a useful aid in monitoring the
reduction progress of your data. This window may be started by
clicking {\tt Workflow -> Runtime Window} from the {\tt Reflex} canvas
menu. You will notice that on the left-hand side the runtime window
has buttons allowing the control of the workflow (\fbox{\tt Go},
\fbox{\tt Pause}, \fbox{\tt Resume}, \fbox{\tt Stop}) and text boxes
for controlling workflow parameters such as the working data directory
etc.

On the right-hand side of the runtime window is a text box with the
title ``Recipe Status'' which lists the current status of each
pipeline recipe and the reduction status of each DataSet. A recipe may
have the following status values:
\begin{itemize}
\item{{\tt Not Running} - The recipe has not yet been run for any DataSet so far.}
\item{{\tt Executing} - The recipe is currently executing for a DataSet.}
\item{{\tt Done} - The last execution of the pipeline recipe was successful.}
\item{{\tt Failed} - The recipe failed on the last DataSet.}
\item{{\tt Skip} - The recipe was skipped for the last DataSet.}
\item{{\tt Disabled} - The recipe was disabled for the last DataSet.}
\item{{\tt Stopped} - The workflow was stopped during the reduction of a DataSet.}
\end{itemize}

Below the list of recipe status values is a detailed list of input and
output files used for each DataSet within each recipe execution. This
information is sometimes very useful for the user who wants to know
exactly which files were used as input for a particular DataSet for a
given pipeline recipe, and where the relevant output files were
written.
 %This file is in the pipedoc directory

%\input{reflex_reducing_own_data} %This file is in the pipedoc directory
%\input{reflex_data_organizer} %This file is in the pipedoc directory
%\input{reflex_dataset_chooser} %This file is in the pipedoc directory
%\input{reflex_product_explorer} %This file is in the pipedoc directory

\subsection{Workflow Actors}
\subsubsection{Simple Actors \label{sec:simple_actors}}

Simple actors have workflow symbols that consist of a single (rather than multiple) green-blue rectangle. They may also have a logo within the rectangle
to aid in their identification. The following actors are simple actors:
\begin{itemize}
\item{\includegraphics[width=2.5cm,height=1.3cm]{reflex_data_organiser_actor.png} - The {\tt Data Organiser} actor.}
\item{\includegraphics[width=2.5cm,height=1.6cm]{reflex_data_set_chooser_actor.png} - The {\tt Data Set Chooser} actor.}
\item{\includegraphics[width=1.6cm,height=0.8cm]{reflex_fits_router_actor.png} - The {\tt Fits Router} actor}
\item{\includegraphics[width=1.8cm,height=1cm]{reflex_product_renamer_actor.png} - The {\tt Product Renamer} actor.}
\item{\includegraphics[width=2.2cm,height=1cm]{reflex_data_filter_actor.png} - The {\tt Data Filter} actor.}
\end{itemize}

Access to the parameters for a simple actor is achieved by
right-clicking on the actor and selecting {\tt Configure Actor}. This
will open an ``Edit parameters'' window. Note that the {\tt Product Renamer}
actor is a jython script (Java implementation of the Python
interpreter) meant to be customised by the user (by double-clicking on
it).

 %This file is in the pipedoc directory

There are no composite actors in the current workflow
and each box corresponds to a single recipe call.

\subsubsection{Lazy Mode \label{sec:lazy_mode}}

By default, all recipe executer actors in a pipeline workflow are
``Lazy Mode'' enabled. This means that when the workflow attempts to
execute such an actor, the actor will check whether the relevant
pipeline recipe has already been executed with the same input files
and with the same recipe parameters. If this is the case, then the
actor will not execute the pipeline recipe, and instead it will simply
broadcast the previously generated products to the output port. The
purpose of the Lazy mode is therefore to minimise any reprocessing of
data by avoiding data rereduction where it is not necessary.
  
One should note that the actor Lazy mode depends on the contents of
the directory specified by \\ {\tt BOOKKEEPING\_DIR} and the relevant
FITS file checksums. Any modification to the directory contents and/or
the file checksums will cause the corresponding actor when executed to
run the pipeline recipe again, thereby rereducing the input data.

The forced rereduction of data at each execution may of course be
desirable. To force a rereduction of all data for all {\tt RecipeExecuter}
actors in the workflow (i.e. to disable Lazy mode for the whole
workflow), set the {\tt EraseDirs} parameter under the ``Global
Parameters'' area of the workflow canvas to {\tt true}. This will then
remove all previous results as well.  To force a rereduction of data
for any single  {\tt RecipeExecuter} actor in the workflow (which will be
inside the relevant composite actor), right-click the  {\tt RecipeExecuter} actor, select
{\tt Configure Actor}, and uncheck the Lazy mode parameter tick-box in
the ``Edit parameters'' window that is displayed.

 %This file is in the pipedoc directory

%Add here a description of the workflow steps like Data organisation,
%routing, creation of calibration files and science reduction.

%Add here a description on how to optimise the results of the workflow.

\section{Frequently Asked Questions}

\begin{itemize}
   \item {\bf Where are my intermediate pipeline products?}
   Intermediate pipeline products are stored in the directory {\tt
   \verb|<|TMP\_PRODUCTS\_DIR\verb|>|} (defined on the workflow canvas)
   and organised further in directories by pipeline recipe.

%\item {\bf I have many DataSets in my data directory. How can I reduce
%  them interactively without having to wait a long time between
%  interactive windows being displayed?}

%Reduce all the DataSets at once with the interactive windows disabled
%for all interactive actors. When this reduction has finished, you
%should re-enable the interactive windows that you require, and run the
%workflow again. The workflow will run in Lazy mode and no time will be
%spent on pipeline reductions, unless you specifically change a
%parameter in one of the interactive windows.
        
%      Note that Lazy mode will not work if the workflow parameter {\tt
%        EraseDirs} is set to {\tt true}.

   \item {\bf Can I use different sets of bias frames to calibrate my
          flat frames and science data?}
   Yes. In fact this is what is currently implemented in the workflow(s).
   Each file in a DataSet has a purpose attached to it (\cite{REFLEXMAN}).
   It is this purpose that is used by the workflow to send the correct
   set of bias frames to the recipes for flat frame combination and 
   science frame reduction, which may or may not be the same set of bias 
   frames in each case.

   \item {\bf Can I launch {\tt Reflex} from the command line?}
   Yes, use the command:
      \begin{verbatim}
      reflex -- -runwf -nocache -nogui <workflow_path>/<workflow>.xml
      \end{verbatim}
   Note that this mode is not fully supported, and the user should be 
   aware of two points. Firstly, the execution prompt is not returned after the
   workflow finishes, and therefore {\tt Reflex} must be manually killed.
   Secondly, all the interactive windows will still appear (if activated in the
   workflow), so it is not suitable for batch processing.
        
   \item {\bf How can I add new actors to an existing workflow?}
   You can drag and drop the actors in the menu on the left of the {\tt Reflex} 
   canvas. Under {\tt Eso-reflex -> Workflow} you may find all the actors
   relevant for pipeline workflows, with the exception of the recipe executer.
   This actor must be manually instantiated using
   {\tt Tools -> Instantiate Component}. Fill in the ``Class name'' field with 
   {\tt org.eso.RecipeExecuter} and in the pop-up window choose the required 
   recipe from the pull-down menu. To connect the ports of the actor, click on
   the source port, holding down the left mouse button, and release the mouse
   button over the destination port. Please consult the Reflex User Manual
   (\cite{REFLEXMAN}) for more information.

   \item {\bf How can I broadcast a result to different subsequent actors?}
   If the output port is a multi-port (filled in white), then you may have
   several relations from the port. However, if the port is a single port
   (filled in black), then you may use the black diamond from the toolbar.
   Make a relation from the output port to the diamond. Then make relations 
   from the input ports to the diamond. Please note that you cannot click to 
   start a relation from the diamond itself. Please consult the Reflex User 
   Manual (\cite{REFLEXMAN}) for more information.

   \item {\bf How can I run manually the recipes executed by Reflex?}
   If a user wants to re-run a recipe on the command line he/she has to go to
   the appropriate reflex\_book\_keeping directory, which is generally 
   reflex\_book\_keeping/\instname/<recipe\_name>\_<number> (for instance 
   reflex\_book\_keeping/\instname/bias\_1/). There, subdirectories exist with 
   the time stamp of the recipe execution (e.g. 2013-01-25T12:33:53.926/). 
   If the user wants to re-execute the most recent processing he/she should 
   go to the {\tt latest} directory and then execute 
   {\tt ESOREX\_CONFIG="REFLEX\_INST/etc/esorex.rc 
   REFLEX\_INST/bin/esorex --recipe-config=<recipe>.rc <recipe> data.sof}, 
   where REFLEX\_INST is the directory where Reflex and the pipelines were
   installed. If the user knows the name of the input raw files for the 
   recipe, the correct directory among the many time stamps can be found via 
   {\tt grep <raw\_file> */data.sof}. Afterwards the procedure is the same 
   as before. The products will appear in the directory from which the recipe
   is called, and not in the reflex\_tmp\_products or reflex\_end\_products
   directory, and they will not be renamed. 

\end{itemize}


%Add here a section on troubleshooting problems.
\section{Troubleshooting}

\putgraph{12.5cm}{NoDataSet.png}{NoDataSet}
{TheDataOrganizer interactive window reports an error 
  ``:No DataSets have been created, check the data set and the OCA rules.''}
\begin{enumerate}
\item  {\bf I downloaded the data from the ESO archive, put them into
  a new directory, tried to run {\tt Reflex} on them, but}
\begin{enumerate}
\item {\bf it crashes}

The current release of the FORS pipeline includes some additional data in the static
calibration frames. The recipes would choke if this data is not present. 
However, the ESO archive with CalSelector may associate calibration data which
is old and Reflex will pick the files either from the installed pipeline
static data or from the CalSelector in a non-deterministic way. In order to 
solve the issue, remove the static calibration data downloaded from the archive
(all the files starting with M.FORS2).

This may happen if one of the files was downloaded only partially
(check for a file with the extension {\tt fits.Z.part}. You will have to
download that file again in order to have an uncorrrupted file (and
remove the partial one).

\item {\bf The DataOrganiser fails with the error message 
 ``:No DataSets have been created, check the data set and the OCA rules.''(see
  Figure \ref{fig:NoDataSet}.)}

This error may be due to the fact that the data provided by the ESO
archive are compressed\\ ({\tt <filename>.fits.Z}).  Please 
remember to uncompress the data before running the workflow in 
{\tt Reflex}.\\[2mm]

Also, please remember that each \instrument\ workflow supports only one mode.
For example, if the data consists entirely of polarimetric observations,
but any other workflow is executed, the Data Organiser actor will not
construct any datasets, showing the mentioned error message.
\end{enumerate}

\item {\bf The ``Select DataSets'' window displays my datasets, but
  some/all of them are greyed out. What is going on?}

If a dataset in the ``Select DataSets'' window is greyed out, then it
means that the dataset which was constructed is missing some key
calibration(s) (i.e. the dataset is incomplete). To find out what
calibration(s) are missing from a greyed out dataset, click on the
dataset in question to highlight it in blue, and then click on the
button \fbox{\tt Inspect Highlighted}. The ``Select Frames'' window
that appears will report the category of the calibration data that are missing (e.g. DARK). From this the user has then to
determine the missing raw data (in this case bias frames). If static
calibrations are missing the mechanism unfortunately does not work,
but such data should be found by {\tt reflex} in\\ {\tt
  <install\_directory>/calib/<pipeline\_version>/cal}

%\item {\bf I'm trying to reduce some old FORS2 data with the pipeline, but at
%the moment of running {\tt fors\_calib}, the arc lamp frame is not found.}

%The Reflex workflow with its default OCA rules works only for data
%taken after October 2006, when the calibration plan was changed. To
%process older data, right click on the {\tt DataOrganiser} to configure it,
%and change the OCA rules file to {\tt fors2\_spec\_wkf\_pre2006.oca} (same
%directory as the default OCA file), and  click on \fbox{\tt
%  Commit}.

\end{enumerate}


\bibliography{cr2res_reflex_tutorial}
\end{document}

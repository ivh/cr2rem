
\section{Geometrical distortions models}
\label{DISTORTIONS}

The reduction of <<IIInstrument>> scientific data made by the <<IIInstrument>> pipeline 
is based on a set of predefined models of the optical distortions
affecting the instrument.

In the <<IIInstrument>> pipeline, optical distortions modeling is obtained
by simple polynomial fitting based on known quantities, 
such as celestial coordinates of astrometric stars,
positions of pinholes on a calibration mask, or spectral lines
wavelengths from a catalog, all compared to the positions of detected
features and patterns on the CCD.

Pipeline recipes related to geometrical calibrations
generate a set of IWS configuration files
where the coefficients of the derived polynomials are stored.
This information will be copied, when appropriate, from the IWS configuration
files to the headers of any dataset generated by the <<IIInstrument>>
instrument.

Optical distortions are not expected to remain constant in time.
Small changes are introduced by modifying the orientation of
the instrument within the gravitational field. A progressive
aging of the structure, and possible interventions on the
instrument, may also contribute to long term changes.

For this reason, the <<IIInstrument>> pipeline recipes will use the
distortion models contained in the datasets headers as
``first guesses'', to be used as a starting point
for further improvement of the solutions, adapting them to the
momentary instrument distortions.

``First guesses'' will work as long as they are close
enough to the real distortions, ensuring a safe identification
of the appropriate reference signals contained in the data.
How much is ``close enough'' depends on a large number of
factors, mainly related to the nature of the distortion,
and to the robustness of the pattern matching algorithm of
the involved pipeline recipe. This will be discussed in more
detail in Section \ref{ALGORITHMS}.

\subsection{Polynomial models}
\label{POLY}

The geometrical distortions introduced by the <<IIInstrument>> + UT optics
can be distinguished into \ {\it optical} \ and \ {\it spectral},
\ mirroring the fundamental instrument setups. Each optical and spectral 
distortion is in its turn described by a set of polynomial models. In some
cases the polynomial models encode not just a distortion (intended
as a transformation within the same coordinate system), but a transformation
from a coordinate system to another that may include
also the geometrical distortions.

Here is an overview of the polynomials used to model each distortion:

\begin{description}
  \item {\bf Optical}
  \begin{itemize}
    \item Mask to CCD transformation (MAS2CCD)
    \begin{itemize}
      \item Transformation matrix (scale, shift, rotation)
      \item Two bivariate polynomial fits of the residuals (for the
            $X$ and the $Y$ CCD coordinates)
    \end{itemize}
    \item CCD to Mask transformation (CCD2MAS)
    \begin{itemize}
      \item Transformation matrix (scale, shift, rotation)
      \item Two bivariate polynomial fits of the residuals (for the
            $x$ and the $y$ Mask coordinates)
    \end{itemize}
    \item Sky to CCD distortion (SKY2CCD)
    \begin{itemize}
      \item Bivariate polynomial fit of the residuals of CCD positions
            derived applying the WCS received from the TCS
    \end{itemize}
    \item CCD to Sky distortion (CCD2SKY)
    \begin{itemize}
      \item Inverse of the bivariate polynomial fit modeling the Sky 
            to CCD distortion
    \end{itemize}
  \end{itemize}
\end{description}

During the data reduction process the Sky to CCD distortion model is
converted by the pipeline into the CO matrix standard, used in the SAO WCSTools package [10].

\begin{description}
  \item {\bf Spectral}
  \begin{itemize}
    \item Zero Order Contamination (ZERO)
    \begin{itemize}
      \item Two bivariate polynomials (separately for the
            $X$ and the $Y$ CCD coordinate) of mask coordinates \ {\it vs}
            \ CCD positions
    \end{itemize}
    \item Optical Distortion (OPT)
    \begin{itemize}
      \item Two bivariate polynomials (separately for the
            $X$ and the $Y$ CCD coordinate) of mask coordinates \ {\it vs}
            \ CCD positions
    \end{itemize}
    \item Spectral Curvature (CRV)
    \begin{itemize}
      \item {\it Local CRV}: Simple polynomial fits of local curvatures
      \item {\it Global CRV}: Bivariate polynomial fits of the coefficients 
            of local CRV \ {\it vs}
            \ CCD positions
    \end{itemize}
    \item Inverse Dispersion Solution (IDS)
    \begin{itemize}
      \item {\it Local IDS}: Simple polynomial fits of wavelengths \ {\it vs} 
            \ CCD positions
      \item {\it Global IDS}: Bivariate polynomial fits of the coefficients 
            of local IDS \ {\it vs}
            \ CCD positions
    \end{itemize}
  \end{itemize}
\end{description}

The so-called ``optical distortion model'' is really a transformation
from Mask to CCD coordinates valid for the spectral instrument setup, 
that includes the optical distortions at a conventional reference wavelength.
The choice of a reference wavelength $\lambda_o$ is in principle arbitrary,
being just a conventional zero-point for all the spectral distortion
models and transformations. In practice $\lambda_o$ is chosen
roughly in the middle of the valid spectral range of a given grism,
possibly matching the wavelength of a bright and isolated line
of the arc lamp catalog used for spectral calibrations.

The coefficients of all the polynomial fits are written to the appropriate
<<IIInstrument>> IWS configuration files, with the exception of the 
local CRV and IDS, whose coefficients are written instead into another pipeline
product, the {\it extraction table} (see Section \ref{UCALDISP}, page
\pageref{EXTRATAB}) in the case of MOS observations, and into the trace 
and IDS tables (see Section \ref{UIFUCALIB}, pages \pageref{IFUTRACE}
and \pageref{IFUIDS}) in the case of IFU observations.

Details on the algorithms applied 
by the relevant pipeline recipes can be found in Section \ref{ALGORITHMS}.
In the present section just a description of the geometrical distortion
models is given.

\newpage

\subsection{Optical distortions}
\label{OPTDIS}

We include in this section any transformation between different
coordinate systems while the instrument is configured in direct 
imaging mode.

Three fundamental coordinate systems can be considered:

\begin{itemize}
\item Celestial (Sky)
\item Telescope focal plane (Mask)
\item Instrument focal plane (CCD).
\end{itemize}

Only the transformations from CCD to Sky and from CCD to Mask (together
with their inversions) are used and supported by the <<IIInstrument>> pipeline.

\subsubsection{CCD to Mask transformation and its inverse}
\label{CCD2MAS}

The transformation from CCD to Mask coordinates
is described by a two-layer model, consisting of a
linear transformation containing rotation, shift, and scaling,
to which a bivariate polynomial fit of the residuals is added.

The linear transformation can be expressed in the form

$$ \cases
{
                x = a_{xx} X + a_{xy} Y + x_o  \cr
                y = a_{yx} X + a_{yy} Y + y_o  \cr
}
$$

where $(X,Y)$ are CCD coordinates (pixels), and $(x,y)$ the 
corresponding mask coordinates (millimetres).

If the mask were perfectly aligned with the CCD, only the
diagonal elements of the matrix, $a_{xx}$ and $a_{yy}$, would
differ from zero, and they would correspond to the scale factor
between mask and CCD (about 0.119 mm/pixel).

The coefficients of the linear transformation for quadrant 
\ $q$ \ are written to the 
entries of the \ {\tt IMG\_mask2ccd\_}$q${\tt.cmf} \ IWS configuration 
file indicated in Table \ref{TCCD2MAS}.

\begin{table}[h]
  \begin{center}
    \begin{tabular}{|c|c|}
    \hline
      \multicolumn{2}{|c|}{\bf CCD to Mask linear transformation} \\
    \hline
      {\tt IMG\_mask2ccd\_}$q${\tt.cmf} & coefficient \\
    \hline
      {\tt PRO CCD MASK X0} & $x_o$ \\
      {\tt PRO CCD MASK XX} & $a_{xx}$ \\
      {\tt PRO CCD MASK XY} & $a_{xy}$ \\
      {\tt PRO CCD MASK Y0} & $y_o$ \\
      {\tt PRO CCD MASK YY} & $a_{yy}$ \\
      {\tt PRO CCD MASK YX} & $a_{yx}$ \\
    \hline
    \end{tabular}
    \caption{\it CCD to Mask linear transformation coefficients.}
    \label{TCCD2MAS}
  \end{center}
\end{table}

The residuals to this linear transformation are modeled by a bivariate
polynomial, that accounts for the optical distortions of the instrument:

$$ \cases
{
\Delta x = \sum_{i,j} x_{ij} X^i Y^j  
                        & with $0 \leq i \leq m , 0 \leq j \leq m$ \cr
\Delta y = \sum_{i,j} y_{ij} X^i Y^j  
                        & with $0 \leq i \leq m , 0 \leq j \leq m$ \cr
}
$$

The coefficients of the distortion for quadrant $q$, and the max degree 
of each variable of the bivariate polynomial, are written to the
entries of the \ {\tt IMG\_mask2ccd\_}$q${\tt.cmf} \ IWS configuration 
file indicated in Table \ref{TTCCD2MAS}.

\begin{table}[h]
  \begin{center}
    \begin{tabular}{|c|c|}
    \hline
      \multicolumn{2}{|c|}{\bf CCD to Mask distortion model} \\
    \hline
      {\tt IMG\_mask2ccd\_}$q${\tt.cmf} & coefficient \\
    \hline
      {\tt PRO CCD MASK XORD} & $m$ \\
      {\tt PRO CCD MASK YORD} & $m$ \\
      {\tt PRO CCD MASK X\_}$i${\tt\_}$j$ & $x_{ij}$ \\
      {\tt PRO CCD MASK Y\_}$i${\tt\_}$j$ & $y_{ij}$ \\
    \hline
    \end{tabular}
    \caption{\it CCD to Mask distortion model coefficients.}
    \label{TTCCD2MAS}
  \end{center}
\end{table}

Currently $m$ must be kept equal to $3$, for compatibility with the VMMPS.
The complete transformation from CCD to Mask is given by the sum of
the linear transformation with the distortion model.

The RMS (in millimetres) of the residuals of the complete transformation is 
also written to the IWS configuration file, at the entries 
\ {\tt PRO CCD MASK XRMS} \ and 
\ {\tt PRO CCD MASK YRMS}, \ together with 
the assigned temperature and time tag,
written to \ {\tt PRO CCD MASK TEMP} \ and \ {\tt PRO CCD MASK DAYTIM}.

The inverse transformation, from Mask to CCD, is completely analogous
to the CCD to Mask transformation.

\subsubsection{CCD to Sky distortion and its inverse}
\label{CCD2SKY}

For transforming CCD pixel coordinates to Sky coordinates and back, a 
WCS is written by the TCS to the FITS header of the observation data.
This transformation is performed by the pipeline calling the
appropriate functions of the SAO WCSTools package [10].

Once a WCS is established, the contribution of the optical distortions
needs to be modeled. This is a distortion, meaning that the transformation 
is performed within the same coordinate system (in this case, the CCD).
It is modeled by a two-branches bivariate polynomial analogous to the one 
used for the Mask to CCD transformations:

$$ \cases
{
X_v = \sum_{i,j} \alpha_{ij} X^i Y^j  
                        & with $0 \leq i \leq m , 0 \leq j \leq m$ \cr
Y_v = \sum_{i,j} \beta_{ij} X^i Y^j  
                        & with $0 \leq i \leq m , 0 \leq j \leq m$ \cr
}
$$

We describe here for simplicity just the CCD to Sky model.
This model is not converting image pixels into celestial coordinates
(RA and Dec), but converts pixel positions $(X,Y)$ on the CCD into 
virtual pixel positions $(X_v,Y_v)$, that are corrected for distortions and
temperature effects. These virtual pixel positions can then be converted
into celestial coordinates using the WCS information present in the 
data header (see Figure \ref{fig:IMSKY}). 

\putgraph{9.5}{skyccd.png}{IMSKY}{Transformations and distortions between sky and CCD.}

The coefficients of the distortion for quadrant $q$, and the max degree 
of each variable of the bivariate polynomial, are written to the
entries of the \ {\tt IMG\_sky2ccd\_}$q${\tt.cmf} \ IWS configuration 
file indicated in Table \ref{TCCD2SKY}.

\begin{table}[h]
  \begin{center}
    \begin{tabular}{|c|c|}
    \hline
      \multicolumn{2}{|c|}{\bf CCD to Sky distortion model} \\
    \hline
      {\tt IMG\_sky2ccd\_}$q${\tt.cmf} & coefficient \\
    \hline
      {\tt PRO CCD SKY XORD} & $m$ \\
      {\tt PRO CCD SKY YORD} & $m$ \\
      {\tt PRO CCD SKY X\_}$i${\tt\_}$j$ & $\alpha_{ij}$ \\
      {\tt PRO CCD SKY Y\_}$i${\tt\_}$j$ & $\beta_{ij}$ \\
    \hline
    \end{tabular}
    \caption{\it CCD to Sky distortion model coefficients.}
    \label{TCCD2SKY}
  \end{center}
\end{table}

For $m$ a value of $3$ is currently chosen.

The RMS of the residuals of the models are also
written to the IWS configuration file, at the entries \ {\tt PRO CCD SKY XRMS} \ and
\ {\tt PRO CCD SKY YRMS}, \ together with the assigned temperature and time tag,
written to \ {\tt PRO CCD SKY TEMP} \ and \ {\tt PRO CCD SKY DAYTIM}.

The inverse model would simply produce the $(X,Y)$ coordinates of the real CCD
from the $(X_v,Y_v)$ virtual coordinates obtained by applying the WCS to
(RA, Dec) positions.

The pipeline converts these distortion models into the CO-matrix convention
that is then written to the FITS headers of the reduced science images.



%%% Local Variables:
%%% mode: latex
%%% TeX-master: t
%%% End:


\section{Frequently Asked Questions}

\begin{itemize}
   \item {\bf Where are my intermediate pipeline products?}
   Intermediate pipeline products are stored in the directory {\tt
   \verb|<|TMP\_PRODUCTS\_DIR\verb|>|} (defined on the workflow canvas)
   and organised further in directories by pipeline recipe.

%\item {\bf I have many DataSets in my data directory. How can I reduce
%  them interactively without having to wait a long time between
%  interactive windows being displayed?}

%Reduce all the DataSets at once with the interactive windows disabled
%for all interactive actors. When this reduction has finished, you
%should re-enable the interactive windows that you require, and run the
%workflow again. The workflow will run in Lazy mode and no time will be
%spent on pipeline reductions, unless you specifically change a
%parameter in one of the interactive windows.
        
%      Note that Lazy mode will not work if the workflow parameter {\tt
%        EraseDirs} is set to {\tt true}.

   \item {\bf Can I use different sets of bias frames to calibrate my
          flat frames and science data?}
   Yes. In fact this is what is currently implemented in the workflow(s).
   Each file in a DataSet has a purpose attached to it (\cite{REFLEXMAN}).
   It is this purpose that is used by the workflow to send the correct
   set of bias frames to the recipes for flat frame combination and 
   science frame reduction, which may or may not be the same set of bias 
   frames in each case.

   \item {\bf Can I launch {\tt Reflex} from the command line?}
   Yes, use the command:
      \begin{verbatim}
      reflex -- -runwf -nocache -nogui <workflow_path>/<workflow>.xml
      \end{verbatim}
   Note that this mode is not fully supported, and the user should be 
   aware of two points. Firstly, the execution prompt is not returned after the
   workflow finishes, and therefore {\tt Reflex} must be manually killed.
   Secondly, all the interactive windows will still appear (if activated in the
   workflow), so it is not suitable for batch processing.
        
   \item {\bf How can I add new actors to an existing workflow?}
   You can drag and drop the actors in the menu on the left of the {\tt Reflex} 
   canvas. Under {\tt Eso-reflex -> Workflow} you may find all the actors
   relevant for pipeline workflows, with the exception of the recipe executer.
   This actor must be manually instantiated using
   {\tt Tools -> Instantiate Component}. Fill in the ``Class name'' field with 
   {\tt org.eso.RecipeExecuter} and in the pop-up window choose the required 
   recipe from the pull-down menu. To connect the ports of the actor, click on
   the source port, holding down the left mouse button, and release the mouse
   button over the destination port. Please consult the Reflex User Manual
   (\cite{REFLEXMAN}) for more information.

   \item {\bf How can I broadcast a result to different subsequent actors?}
   If the output port is a multi-port (filled in white), then you may have
   several relations from the port. However, if the port is a single port
   (filled in black), then you may use the black diamond from the toolbar.
   Make a relation from the output port to the diamond. Then make relations 
   from the input ports to the diamond. Please note that you cannot click to 
   start a relation from the diamond itself. Please consult the Reflex User 
   Manual (\cite{REFLEXMAN}) for more information.

   \item {\bf How can I run manually the recipes executed by Reflex?}
   If a user wants to re-run a recipe on the command line he/she has to go to
   the appropriate reflex\_book\_keeping directory, which is generally 
   reflex\_book\_keeping/\instname/<recipe\_name>\_<number> (for instance 
   reflex\_book\_keeping/\instname/bias\_1/). There, subdirectories exist with 
   the time stamp of the recipe execution (e.g. 2013-01-25T12:33:53.926/). 
   If the user wants to re-execute the most recent processing he/she should 
   go to the {\tt latest} directory and then execute 
   {\tt ESOREX\_CONFIG="REFLEX\_INST/etc/esorex.rc 
   REFLEX\_INST/bin/esorex --recipe-config=<recipe>.rc <recipe> data.sof}, 
   where REFLEX\_INST is the directory where Reflex and the pipelines were
   installed. If the user knows the name of the input raw files for the 
   recipe, the correct directory among the many time stamps can be found via 
   {\tt grep <raw\_file> */data.sof}. Afterwards the procedure is the same 
   as before. The products will appear in the directory from which the recipe
   is called, and not in the reflex\_tmp\_products or reflex\_end\_products
   directory, and they will not be renamed. 

\end{itemize}

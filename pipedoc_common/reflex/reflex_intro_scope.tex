\section{Introduction And Scope}

{\tt Reflex}\footnote{User support for this software is available by sending 
enquiries to \mail{usd-help@eso.org}} is the ESO Recipe Flexible Execution 
Workbench, an environment to run ESO VLT pipelines which employs a workflow 
engine (Kepler\footnote{\http{kepler-project.org}}) to provide a real-time
visual representation of a data reduction cascade, called a workflow,
which can be easily understood by most astronomers.  This document is
a tutorial designed to enable the user to employ the \instname\, workflow to
reduce his/her data in a user-friendly way, concentrating on
high-level issues such as data reduction quality and signal-to-noise 
(S/N) optimisation.

A workflow accepts science and calibration data, as delivered to PIs
in the form of PI-Packs (until October 2011) or downloaded from the
archive using the CalSelector
tool\footnote{\http{www.eso.org/sci/archive/calselectorInfo.html}} and
organises them into DataSets, where each DataSet contains one
science object observation (possibly consisting of several science
files) and all associated raw and static calibrations required for a
successful data reduction. The data organisation process is fully
automatic, which is a major time-saving feature provided by the
software. The DataSets selected by the user for reduction are fed
through the workflow which executes the relevant pipeline recipes (or
stages) in the correct order.
%, providing optional user interactivity at
%key data reduction points with the aim of enabling the iteration of
%certain recipes in order to obtain better results. 
Full control of the
various recipe parameters is available within the workflow, and the
workflow deals automatically with optional recipe inputs via built-in
conditional branches. Additionally, the workflow stores the reduced
final data products in a logically organised directory structure and
employing user-configurable file names.

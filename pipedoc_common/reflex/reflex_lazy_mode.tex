\subsubsection{Lazy Mode \label{sec:lazy_mode}}

By default, all recipe executer actors in a pipeline workflow are
``Lazy Mode'' enabled. This means that when the workflow attempts to
execute such an actor, the actor will check whether the relevant
pipeline recipe has already been executed with the same input files
and with the same recipe parameters. If this is the case, then the
actor will not execute the pipeline recipe, and instead it will simply
broadcast the previously generated products to the output port. The
purpose of the Lazy mode is therefore to minimise any reprocessing of
data by avoiding data rereduction where it is not necessary.
  
One should note that the actor Lazy mode depends on the contents of
the directory specified by \\ {\tt BOOKKEEPING\_DIR} and the relevant
FITS file checksums. Any modification to the directory contents and/or
the file checksums will cause the corresponding actor when executed to
run the pipeline recipe again, thereby rereducing the input data.

The forced rereduction of data at each execution may of course be
desirable. To force a rereduction of all data for all {\tt RecipeExecuter}
actors in the workflow (i.e. to disable Lazy mode for the whole
workflow), set the {\tt EraseDirs} parameter under the ``Global
Parameters'' area of the workflow canvas to {\tt true}. This will then
remove all previous results as well.  To force a rereduction of data
for any single  {\tt RecipeExecuter} actor in the workflow (which will be
inside the relevant composite actor), right-click the  {\tt RecipeExecuter} actor, select
{\tt Configure Actor}, and uncheck the Lazy mode parameter tick-box in
the ``Edit parameters'' window that is displayed.


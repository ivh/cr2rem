\section{The \instname\, Workflow \label{sec:wkf_general_desc}}

The \instname\,  workflow canvas is organised into a number of areas.
From top-left to top-right you
will find general workflow instructions, directory parameters, and
global parameters.  In the middle row you will find five boxes
describing the workflow general processing steps in order from left to
right, and below this the workflow actors themselves are organised
following the workflow general steps. 

\subsection{Workflow Canvas Parameters \label{sec:wkf_canvpar}}

The workflow canvas displays a number of parameters that may be set by
the user. Under ``Setup
Directories'' the user is only required to set the {\tt
  ROOT\_DATA\_DIR} to the working directory for the DataSet(s) to be
reduced, which, by default, is set to the directory containing the
demo data. Raw data should be stored in a subdirectory of {\tt
  ROOT\_DATA\_DIR}, defined by the parameter {\tt RAWDATA\_DIR}, which
is recursively scanned by the {\tt Data Organiser} actor for input raw
data. The directory {\tt CALIB\_DATA\_DIR}, which is within the
pipeline installation directory, is also scanned by the {\tt Data
  Organiser} actor to find any static calibrations that may be missing
in your DataSet(s).  If required, the user may edit the directories
{\tt BOOKKEEPING\_DIR}, {\tt LOGS\_DIR}, {\tt TMP\_PRODUCTS\_DIR}, and
{\tt END\_PRODUCTS\_DIR}, which correspond to the directories where
book-keeping files, logs, temporary products and end products are
stored, respectively (see the Reflex User Manual for further details;
\cite{REFLEXMAN}).

Under the ``Global Parameters'' area of the workflow canvas, the user
may set the {\tt FITS\_VIEWER} parameter to the command used for
running his/her favourite application for inspecting FITS
files. Currently this is set by default to {\tt fv}, but other
applications, such as {\tt ds9}, {\tt skycat} and {\tt gaia} for
example, may be useful for inspecting image data.

By default the {\tt EraseDirs} parameter is set to {\tt false}, which
means that no directories are cleaned before executing the workflow,
and the recipe actors will work in Lazy mode (see
Section~\ref{sec:lazy_mode}), reusing the previous pipeline recipe outputs
where input files and parameters are the same as for the previous
execution, which saves considerable processing time. Sometimes it is
desirable to set the {\tt EraseDirs} parameter to {\tt true}, which
forces the workflow to recursively delete the contents of the
directories specified by {\tt BOOKKEEPING\_DIR}, {\tt LOGS\_DIR}, and
{\tt TMP\_PRODUCTS\_DIR}. This is useful for keeping disk space usage
to a minimum and will force the workflow to fully rereduce the data
each time the workflow is run.



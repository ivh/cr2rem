\section{Overview}
\label{OVER}

In collaboration with instrument consortia, the Data Flow Systems
Department (DFS) of the Data Management and Operation Division is
implementing data reduction pipelines for the most commonly used VLT/VLTI
instrument modes. These data reduction pipelines have the following three
main purposes:

\begin{description}
\item [Data quality control:] pipelines are used to produce the
quantitative information necessary to monitor instrument performance.
\item [Master calibration product creation:] pipelines are used to
produce master calibration products ({\it e.g.}, combined bias frames,
super-flats, wavelength dispersion solutions).
\item [Science product creation:] using pipeline-generated master
calibration products, science products are produced for the supported
instrument modes ({\it e.g.}, combined ISAAC jitter stacks; bias-corrected,
flat-fielded FORS images, wavelength-calibrated UVES spectra). The
accuracy of the science products is limited by the quality of the
available master calibration products and by the algorithmic
implementation of the pipelines themselves. In particular, adopted
automatic reduction strategies may not be suitable or optimal for all
scientific goals.
\end{description}

Instrument pipelines consist of a set of data processing modules that can be
called from the command line, from the automatic data management tools 
available on Paranal or from Gasgano.



ESO offers two front-end applications for launching pipeline
recipes, \ {\it Gasgano} [14] \ and \ {\it EsoRex}, both included 
in the pipeline distribution (see Appendix \ref{installation}, 
page \pageref{installation}). These applications can also be downloaded 
separately from \ \texttt{http://www.eso.org/gasgano} 
\ and \ \texttt{http://www.eso.org/cpl/esorex.html}.
An illustrated introduction to Gasgano is provided in the "Quick Start" 
Section of this manual (see page \pageref{COOK}).


%A short overview of the instrument and raw data are provided in sections 
%\ref{instrument} and \ref{rawdata}. Auxiliary data are described in 
%section \ref{ancillary_data}. An overview of the data reduction, what 
%are the input data, and the recipes involved in the calibration cascade 
%together with a brief description on how to execute each recipe is 
%provided in section \ref{data_reduction}.
%More details on what is the input to each recipe, its products and quality
%control parameters, and their input controlling parameters is given 
%in section \ref{pipeline_recipe_interfaces}.
%A detailed description of the recipe algorithm used and on how each recipe 
%is implemented is provided in section \ref{algorithms}.
%To support the IIINSTRUMENT package installation we also provide 
%Appendix \ref{installation}.


The IIINSTRUMENT instrument and the different types of IIINSTRUMENT raw frames 
and auxilliary data are described in Sections \ref{instrument}, \ref{DATA}, and \ref{CALDATA}.


A brief introduction to the usage of the available reduction recipes 
using Gasgano or EsoRex is presented in Section \ref{COOK}.
In section \ref{PROBLEMS} we advice the user 
about known data reduction problems providing also possible solutions.

An overview of the data reduction, what 
are the input data, and the recipes involved in the calibration cascade 
is provided in section \ref{data_reduction}.


More details on what are inputs, products, quality control measured quantities,
 and controlling parameters of each recipe is given 
in section \ref{pipeline_recipe_interfaces}.

More detailed descriptions of the data reduction algorithms used by
the individual pipeline recipes can be found in Section \ref{ALGORITHMS}.

%\framebox{
%\parbox{\linewidth}{
%In Section \ref{WHATSNEW} an overview of what's new on IIINSTRUMENT Pipeline
%release 2.0 is given.
%}}

In Appendix \ref{installation} the installation of the IIINSTRUMENT pipeline 
recipes is described and 
in Appendix \ref{ABBREVIATIONS} a list of used abbreviations and acronyms 
is given.

%%% Local Variables: 
%%% mode: latex
%%% TeX-master: t
%%% End: 
\section{<<IIInstrument>> Instrument Description}
\label{INSTR}

<<IIInstrument>> has been developed under ESO contract by the VIRMOS Consortium, 
headed by the Laboratoire d'Astrophysique de Marseille. 

The instrument has been made available to the community and started 
operations in Paranal on April $1^{st}$, 2003.

In this chapter a brief description of the <<IIInstrument>> instrument is given.
A more complete documentation can be found in the <<IIInstrument>> User Manual,
downloadable from \ {\tt http://www.eso.org/instruments/vimos/}

\subsection{Overview}
\label{INSTR:OVER}

<<IIInstrument>> is aimed at survey-type programs with emphasis on large object samples 
rather than individual objects. <<IIInstrument>> is designed for Wide Field
Imaging (14' x 16') and extremely high Multi Object Spectroscopy 
capability (up to several hundred slits). In addition, it has a 
unique {\it Integral Field Unit} (IFU) providing a field-of-view 
up to 1 arc minute at 0.67''/fibre in low resolution spectroscopy. 

The field-of-view is split in four identical channels. Field lenses 
provide a corrected telescope focal plane where flat masks are 
inserted in MOS mode. For the IFU instrument mode a special
mask bearing the IFU pseudo-slits is used.
Pupil relay lenses, folding mirrors and collimators 
direct the light to the four cameras.
Grisms are inserted in front of the cameras in spectroscopic mode.
The detectors are four 2k x 4k EEV CCDs with pixel size 15 $\mu$ 
(see \ {\tt http://www.eso.org/projects/odt/Vimos/vimos.html} \ for 
detector design and performance reports of the four <<IIInstrument>> 
CCD systems).

\subsection{Direct imaging}

The field-of-view consists of 4 quadrants of 7' x 8' each separated 
by a cross 2' wide, with a sampling of 0.205''/pixel.

The available filters, U, B, V, R, I, and z, are close to the Mould 
definition, and allow 
to minimise the colour terms to transform to the Johnson system.

The filter transmission curves are available from  \
{\tt http://www.eso.org/instruments/vimos}.

\subsection{Multi-Object-Spectroscopy (MOS)}
\label{INSTR:MOS}

The multi-object mode of <<IIInstrument>> uses grisms and masks. ESO distributes
the {\it <<IIInstrument>> Mask Preparation Software} (VMMPS), a package
developed by the VIRMOS Consortium for slit definition and 
positioning on a preliminary exposure on the sky region to be observed. 
The user can define rectangular, curved or inclined slits of widths
larger than 0.4''.
The masks are laser cut in INVAR plates on Paranal with the 
{\it Mask Manufacturing Unit} (MMU). 
The instrument cabinet has a capacity of 15 masks, some of which 
are meant for maintenance and calibration purposes\footnote{For 
technical reasons only 7 slots of the masks cabinet
are currently used.}. 

There are 6 grisms available, all operating in first order.
Their spectral characteristics are given in Table \ref{TGRISMS}.

\begin{table}[h]
  \begin{center}
    \begin{tabular}{|l|l|c|c|c|c|}
    \hline
      {\bf Grism} & {\bf Filter} & {\bf $\lambda_c$} (\AA) & {\bf $\lambda$ range} (\AA) & {\bf R} & {\bf Dispersion} (\AA/pixel) \\
    \hline

LR\_red    & OS\_red  & 7500 & 5500 - 9500 & 210 - 260 & 7.3 \\
LR\_blue   & OS\_blue & 4800 & 3700 - 6700 & 180 - 220 & 5.3 \\
MR         & GG435    & 7000 & 5000 - 10000 & 580 - 720 & 2.5 \\
HR\_red    & GG475    & 7400 & 5650 - 9500 & 2500 - 3100 & 0.6 \\
HR\_orange & GG435    & 6310 & 4550 - 8400 & 2150 - 2650 & 0.6 \\
HR\_blue   & none     & 5100 & 3650 - 6900 & 2050 - 2550 & 0.5 \\

    \hline
    \end{tabular}
    \caption{\it <<IIInstrument>> grisms. 
$\lambda_c$ is the zero deviation
(or central) wavelength, and $R$ is the spectral resolution for a
1'' MOS slit, corresponding to $\sim 0.8$ IFU fibre. The spectral ranges
are given with the specified filter in.
The HR\_red grism is not available for quadrant 4, and is replaced
in this case with a HR\_orange grism. The transmission curves for
the four grism/filter units are available at \
{\tt http://www.eso.org/instruments/vimos}.}
    \label{TGRISMS}
  \end{center}
\end{table}

With LR grisms, a spectrum will typically span less than 600 pixels 
along the dispersion direction. This allows a multiplexing factor 
of 5, {\it i.e.}, to stack up to five spectra along the dispersion 
direction, provided that there are enough well spaced targets in 
the field-of-view. 

With MR grisms, a spectrum will span about 2000 pixels when used 
with the GG435 filter. It is therefore possible to stack
up to 2 spectra along the dispersion direction, provided that the slits 
are positioned at the very edges of the imaging field-of-view. 

With HR grisms the spectra extend beyond the detector length,
therefore multiplexing is impossible and
the observable spectral interval depends on the position 
of a slit on the mask, spanning about
2400 \AA\ for the HR\_red and HR\_orange grisms, and
about 2000 \AA\ for the HR\_blue grism.

A further constraint on the slit positions comes from the presence of 
the $0^{th}$, $-1^{st}$ and $2^{nd}$ grism diffraction orders.
At low spectral resolution, a dim second order spectrum at twice
the spectral resolution would be included in the CCD in the case of slits
located in the lower ({\it i.e.}, bluer) regions of the mask. This spectrum
would likely contaminate the multiplexed first order spectra on the
red side of the CCD. Similarly, a mirrored $-1^{st}$ order spectrum at the same
resolution of the $1^{st}$ order spectrum and with about 1/6
of its luminosity, would be included
in the CCD in the case of slits from the highest ({\it i.e.}, redder)
regions of the mask. This spectrum
would likely contaminate the multiplexed first order spectra on the
blue side of the CCD (see an illustration of $-1^{st}$ contamination on 
page \pageref{CONTAM}).
For this reason multiplexed slits are constrained to be identical,
and to have the same position along the cross-dispersion direction:
in the assumption of negligible spectral curvatures in all orders, 
the $0^{th}$, $2^{nd}$ and $-1^{st}$ contaminations would then be 
removed by the sky subtraction procedure.

Aside from the above considerations, the number of slits that can be 
accommodated in one mask obviously depends on the target density. 
Simulations on real fields using VMMPS show that above a density 
of about $7\cdot10^4/$degree$^2$ it is possible to define 
masks with up to 160 10'' long slits/quadrant. 
This number drops to 125 slits/quadrant at a density of about 
$4\cdot10^4/$degree$^2$.

\newpage

\subsection{Integral Field Unit (IFU)}
\label{INSTR:IFU}

The <<IIInstrument>> IFU is the largest ever made for such an application. 
It consists of 6400 (80 x 80) fibres, coupled to microlenses. 
The field-of-view is square, with a continuous spatial sampling 
(the dead space between fibres is below 10\% 
of the fibre-to-fibre distance). 
At the entrance of the IFU there is a focal elongator providing 
two spatial samplings of 0.33''/fibre and 0.67''/fibre. 

The fibres are
split into 16 bundles of 400 fibres each. Each instrument 
quadrant receives 4 bundles that are arranged along
4 parallel pseudo-slits providing 4 multiplexed series of 400
spectra each.

The field-of-view is modified according to the used
spectral resolution. At low spectral resolution 
the field is respectively 54'' x 54'' with 
0.67''/fibre, and 27'' x 27'' with 0.33''/fibre, 
80 fibres on a side. All the pseudo-slits are illuminated,
and the multiplexed spectra belonging to different pseudo-slits
would contaminate each other in some measure. For instance,
the second order spectra of a bright object on pseudo-slit 2 
of quadrant 2 would contaminate the spectra on pseudo-slits
3 and 4, creating obvious ghosts in the corresponding regions
of the reconstructed field-of-view (see Figures \ref{IFUSLIT} 
and \ref{IFUHEAD}, pages \pageref{IFUSLIT}--\pageref{IFUHEAD}).

At medium and high resolution just the 4 central bundles on the 
IFU head are illuminated (see Figure \ref{IFUHEAD}, page \pageref{IFUHEAD}). 
Only one pseudo-slit per quadrant is used, since the spectra span 
the whole detector and multiplexing is impossible. 
The field-of-view is therefore 4 times smaller, {\it i.e.}, 27'' x 27'' 
with 0.67''/fibre, and 13'' x 13'' with 0.33''/fibre, 40 fibres on a side. 

The fibre-to-fibre distance at detector level is about 5.0 pixels, 
while the fibre profile FWHM is about 3.2 pixels. 
The spectral resolution is approximately 1.25 times the spectral resolution 
corresponding to a 1'' slit in MOS mode (see Table \ref{TGRISMS}).
The spectral coverage is identical to the MOS case for
LR and MR grisms. For HR grisms the situation is different because
the spectral range is too large to be contained on the CCD,
and since the central slit-of-fibres is shifted by about 140 pixels 
from the chip centre in (spectrally) opposite directions
depending on the instrument quadrant, the usable spectral range
is reduced by about 160 \AA\, leading to Table \ref{TIFU}.

\begin{table}[h]
  \begin{center}
    \begin{tabular}{|l|l|}
    \hline
      {\bf Grism} & {\bf $\lambda$ range (\AA)} \\
    \hline

HR\_red     & 6350 - 8600 \\
HR\_orange  & 5250 - 7550 \\ 
HR\_blue    & 4200 - 6150 \\

    \hline
    \end{tabular}
    \caption{\it <<IIInstrument>> IFU usable spectral range in high spectral resolution
mode.}
    \label{TIFU}
  \end{center}
\end{table}

\newpage 

\subsection{IFU components numbering scheme}
\label{IFUDEF}

The conventions used in the <<IIInstrument>> IFU pipeline
recipes to indicate IFU fibers,
IFU masks and pseudo-slits are described in this section.

\begin{description}
  \item {\bf IFU masks:} <<IIInstrument>> has 4 IFU masks. They are counted as
        the <<IIInstrument>> quadrants to which they correspond, {\it i.e.},
        counterclockwise, with the same convention used in the
        cartesian plane (see Figure \ref{fig:IFUMASK}).

\putgraph{6}{quadrants.png}{IFUMASK}{Counting <<IIInstrument>> quadrants.}

        In spectral mode, blue is down and red is up in all quadrants.

  \item {\bf IFU pseudo-slits:} Each <<IIInstrument>> mask hosts 4 IFU pseudo-slits,
        numbered from 1 to 4. The pseudo-slit 1
        is the one that is somewhat more separated from the other ones
        (see Figure \ref{fig:IFUSLIT}).

\putgraph{11}{pseudoslits.png}{IFUSLIT}{Counting IFU pseudo-slits.}


  \item {\bf IFU fibers:} Each IFU pseudo-slit hosts 400 fibers, divided
        into 5 blocks of 80 fibers each. The fibers are counted from
        1 to 400, always starting from the left.

  \item {\bf IFU head:} Each pseudo-slit corresponds to a 20x20 region of the
        80x80 IFU head (see Figure \ref{fig:IFUHEAD}).

\putgraph{11}{ifuhead.png}{IFUHEAD}{IFU head. The number of the corresponding pseudo-slit is
         indicated within each 20x20 fiber module.}

        North is to the right, and East is up. The exact spatial position
        for each individual fiber is listed in the IFU tables (see Table
        \ref{IFUTABLE}, page \pageref{IFUTABLE}).

\item {\bf Illuminated pseudo-slits:} In LR observations all the
      pseudo-slits are illuminated (multiplexing). In MR and HR
      observations, just the central pseudo-slits (numbered 2)
      are used.

\end{description}

%%% Local Variables:
%%% mode: latex
%%% TeX-master: t
%%% End:


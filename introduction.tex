% "@(#) $Id: introduction.tex,v 1.5 2011-09-09 10:55:58 cgarcia Exp $"
  
\section{Introduction}

\subsection{Purpose}

The \pipename\, pipeline is a subsystem of the 
{\it VLT Data Flow System} (DFS). It is used in two 
operational environements, for the ESO {\it Data Flow Operations} 
(DFO), and for the {\it Paranal Science Operations} (PSO), 
in the quick-look assessment of data, in the generation of master 
calibration data, in the reduction of scientific exposures, 
and in the data quality control. Additionally, the \pipename\, 
pipeline recipes are made public to the user community, to 
allow a more personalised processing of the data from the instrument. 
The purpose of this document is to describe a typical \pipename\, 
data reduction sequence with the \pipename\, pipeline.

This manual is a complete description of the data reduction recipes 
implemented by the the \pipename\, pipeline, reflecting the status 
of the \pipename\, pipeline as of \releasedate\, (version \pipelinevers).

\subsection{Acknowledgements}

\subsection{Scope}

This document describes the \pipename\, pipeline used at ESO-Garching
and ESO-Paranal for the purpose of data assessment and data quality
control. 

Updated versions of the present document may be found on [1].
For general information about the current instrument pipelines status we remind the user of [2].
Quality control information are at [3].

Additional information on the Common Pipeline Library (CPL), ESOREX, HDRL can be found 
respectively at [4], [5], [6]. The Gasgano tool is described in [14]. 
A description of the instrument is in [7]. 
The \pipename\, instrument user manual is in [8] 
while results of Science Verifications (SV) are at [9].

\subsection{Reference and applicable documents}

\bibliography{cr2repdoc}


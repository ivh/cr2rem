% "@(#) $Id: introduction.tex,v 1.5 2011-09-09 10:55:58 cgarcia Exp $"
  
\section{Introduction}

\subsection{Purpose}

The IIINSTRUMENT pipeline is a subsystem of the 
{\it VLT Data Flow System} (DFS). It is used in two 
operational environements, for the ESO {\it Data Flow Operations} 
(DFO), and for the {\it Paranal Science Operations} (PSO), 
in the quick-look assessment of data, in the generation of master 
calibration data, in the reduction of scientific exposures, 
and in the data quality control. Additionally, the IIINSTRUMENT 
pipeline recipes are made public to the user community, to 
allow a more personalised processing of the data from the instrument. 
The purpose of this document is to describe a typical IIINSTRUMENT 
data reduction sequence with the IIINSTRUMENT pipeline.

This manual is a complete description of the data reduction recipes 
implemented by the the IIINSTRUMENT pipeline, reflecting the status 
of the IIINSTRUMENT pipeline as of xx.xx.2006 (version 1.xx.xx).

\subsection{Acknowledgements}

The IIINSTRUMENT pipeline is based on the SPIFFI Data Reduction Software  
developed by the Max-Planck-Institut  f�r extraterrestrische Physik  (MPE).
We would like to thank the SPIFFI team for providing ESO with a  complete 
and efficient data reduction software and for their help in  documenting, 
testing, debugging the recipes and the pipeline during  several 
commissioning and science verifications phases.
We are particularly grateful to the MPE responsibles for the data 
reduction: Jurgen Schreiber, Matthew Horrobin and Roberto Abuter 
for their contributions and support.

This release benefits also from the feedback provided by the IIINSTRUMENT SV team 
and IIINSTRUMENT instrument operations team. 
In particular we would like to thank Wolfram Howard (ESO, 
Data Flow Operations department) to have provided a lot of support
to test the pipeline and several suggestions for improvements. 
Jutta Neumann, an ESO paid associate, very kindly tested the pipeline
and provided valuable advice to improve from a scientific user prospective 
either the software and the documentation.
Useful was the feedback provided by Joachim Hankel, an ESO paid associate,
to improve the documentation.
\subsection{Scope}

This document describes the IIINSTRUMENT pipeline used at ESO-Garching
and ESO-Paranal for the purpose of data assessment and data quality
control. 

Updated versions of the present document may be found on [1].
For general information about the current instrument pipelines status we remind the user of [2].
Quality control information are at [3].

Additional information on QFITS, the Common Pipeline Library (CPL) and ESOREX can be found 
respectively at [4], [5], [6]. The Gasgano tool is described in [14]. 
A description of the instrument is in [7]. 
The IIINSTRUMENT instrument user manual is in [8] 
%the calibration plan in [12] (not yet!) 
while results of Science Verifications (SV) are at [9].


\begin{tabular}[hc]{cll}
\end{tabular}

\subsection{Reference and applicable documents}

\bibliography{iiinstrumentpdoc}


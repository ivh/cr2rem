\section{\label{data_reduction}Data Reduction}

In this section, after an overview of the main problems the data reduction 
needs to solve, we list the required data and the recipes which allow to 
solve them, giving the data reduction sequence necessary to reduce calibration 
and science data. 

\subsection{Data reduction overview}

\begin{itemize}
\item Correct for the detector signature: bad pixels, detector contribution 
to the measured signal, flat fielding (correct pixel to pixel gain variations 
and relative slitlet throughput differences), correct geometric distortions.
\item Perform the wavelength calibration.
\item ...

\end{itemize}
\begin{figure}[ht]
\begin{center}
\begin{tabular}{c}
%{\psfig{figure=principle.ps,width=12truecm}} \\
\end{tabular}
\end{center}
\caption{}
\end{figure}


\subsection{Required input data}
To be able to reduce science data one needs to use raw, product data and 
pipeline recipes in a given sequence which provides all the input necessary 
to each pipeline recipe. We call this sequence a data reduction cascade. 
The \instname\, data reduction cascade involves the following input data:
\begin{itemize}
\item Reference files:
\begin{itemize}
\item ...
\end{itemize}
\item Raw frames:
\begin{itemize}
\item ...
\end{itemize}
\item Calibration data products.
\begin{itemize}
\item ... 
\end{itemize}
\end{itemize}


\subsection{Reduction cascade} 


The \instname\, data reduction follows the following sequence. A short 
description of the available recipes is given in section \ref{RECIPES}.
In parenthesis we provide the value of the DO category corresponding to each 
frame.

%\putgraph{16}{cr2resassocmap.png}{ASSOCMAP}{\instname\, Association Map}                        

    



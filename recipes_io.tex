\section{\label{pipeline_recipe_interfaces}Pipeline Recipes Interfaces} 

In this section we provide for each recipe examples of 
the required input data and 
their tags. In the following  we assume that 
/path\_file\_raw/filename\_raw.fits and
/path\_file\_cdb/filename\_cdb.tfits are existing FITS files
(e.g. /data1/sinfoni/com2/SINFO.2004-08-16T02:54:04.353.fits and
/cal/sinfo/ifu/cal/DISTORTION\_K.tfits).
 
We also provide a list of the pipeline products for each recipe, indicating 
their default recipe name,  
the value of the FITS keyword HIERARCH ESO PRO CATG (in short PRO.CATG) and a 
short description. The relevant keywords are PRO.CATG, used to classify each 
frame, and to associate to each raw frame the proper calibration frame:

\begin{longtable}{|l|l|}
\hline
 Association keyword & Information \\
\hline
HIERARCH ESO INS SETUP ID     & band \\
HIERARCH ESO INS OPTI1 NAME   & Pixel scale      \\
HIERARCH ESO DET DIT          & Integration time  \\
\hline
\end{longtable}


For each recipe we also list in a table the input parameters (as they
appear in the recipe configuration file), the corresponding aliases
(the corresponding names to be eventually set on command line) and their
default values. Also quality control parameters are listed.
Those are stored in relevant pipeline products.
More information on instrument quality control can be found on 
http://www.eso.org/qc

In addition to the products mentioned below, all recipes produce a PAF
 (VLT parameter file) which is an intermediate pipeline data file
 containing quality control parameter values.

We distinguish between recipes involved in the data reduction cascade 
(having prefix si\_rec) and user utilities (with prefix si\_utl).

\subsection{si\_rec\_detlin}

The recipe si\_rec\_detlin computes the detector responsivity as a function
of the pixel intensity and determines when it becomes non linear. 

\subsubsection{Input}
\begin{verbatim}
/path_file_raw/SINFO.2004-08-16T02:54:04.353.fits LINEARITY_LAMP
/path_file_raw/SINFO.2004-08-16T02:53:37.089.fits LINEARITY_LAMP
/path_file_raw/SINFO.2004-08-16T02:52:23.028.fits LINEARITY_LAMP
/path_file_raw/SINFO.2004-08-16T02:51:59.774.fits LINEARITY_LAMP
/path_file_raw/SINFO.2004-08-16T02:50:38.991.fits LINEARITY_LAMP
/path_file_raw/SINFO.2004-08-16T02:50:11.797.fits LINEARITY_LAMP
/path_file_raw/SINFO.2004-08-16T02:49:04.887.fits LINEARITY_LAMP
/path_file_raw/SINFO.2004-08-16T02:48:36.792.fits LINEARITY_LAMP
/path_file_raw/SINFO.2004-08-16T02:47:28.191.fits LINEARITY_LAMP
/path_file_raw/SINFO.2004-08-16T02:47:07.438.fits LINEARITY_LAMP
/path_file_raw/SINFO.2004-08-16T02:45:59.488.fits LINEARITY_LAMP
/path_file_raw/SINFO.2004-08-16T02:45:42.255.fits LINEARITY_LAMP
\end{verbatim}


\subsubsection{Output}

\begin{longtable}{|*3{l|}}
\hline
default recipe file name     & PRO.CATG  & short description  \\
\hline
lin\_det\_info.tfits         & LIN\_DET\_INFO & Table with coefficients of non linear fit \\
                             &                & to median of each flat image image \\
gain\_info.tfits             & GAIN\_INFO     & Table with detector's gain values \\
out\_bplin\_coeffsCube.fits  & BP\_COEFF      & image with coefficients of non linear fit to pixel's \\
                             &                & intensity used to evaluate non linearity \\
out\_bp\_lin.fits            & BP\_MAP\_NL & Non linear bad pixel map \\
\hline
\end{longtable}

\subsubsection{Quality control}

The pipeline computes the non linear coefficients, the number of non linear
pixels per grating, the detector gain. 

\paragraph{Detector non linearity}
The detector non linearity is computed as described in
\ref{detector_non_linearity}. The computed coefficients are
QC.BP-MAP.LINi.MED (i=0,1,2).

A different method is applied to determine the same quantity as described in
\ref{detector_non_linearity2}. The computed coefficients are
QC.BP-MAP.LINi.MEAN (i=0,1,2).

\paragraph{Non linear bad pixels}
The pipeline computes the number of non linear bad pixels. Those are given
by QC.BP-MAP.NBADPIX and are obtained with the method QC.BP-MAP.METHOD.

\paragraph{Detector gain}
The detector gain is computed as described in \ref{gain} and is given by the
value of QC.GAIN.


\subsubsection{Parameters}
\begin{longtable}{|*3{l|}}
\hline
parameter                                 & alias                      & default \\
\hline
sinfoni.bp\_lin.order                     & bp\_lin-order              & 2
     \\
sinfoni.bp\_lin.thresh\_sigma\_factor     & bp\_lin-thresh\_sigma\_fct & 10.0   \\
sinfoni.bp\_lin.nlin\_threshold           & bp\_lin-nlin\_threshold    & 0.5   \\
sinfoni.bp\_lin.low\_rejection            & bp\_lin-lo\_rej            & 10.0  \\
sinfoni.bp\_lin.high\_rejection           & bp\_lin-hi\_rej            & 10.0  \\
\hline
\end{longtable}


\subsection{si\_rec\_mdark}

The recipe si\_rec\_mdark generates a master dark from a set of raw darks by
stacking frames with rejection of outliers. It also generates a bad pixel map 
flagging the hot-current pixels.

\subsubsection{Input}

\begin{verbatim}
/path_file_raw/SINFO.2004-08-16T01:24:53.070.fits DARK
/path_file_raw/SINFO.2004-08-16T01:09:22.905.fits DARK
/path_file_raw/SINFO.2004-08-16T00:53:51.890.fits DARK
/path_file_raw/SINFO.2004-08-16T00:38:14.994.fits DARK
\end{verbatim}


\subsubsection{Output}

\begin{longtable}{|*3{l|}}
\hline
default recipe file name & PRO.CATG      & short description \\
\hline
out\_bp\_noise.fits      & BP\_MAP\_HP   & bad pixel map, method=''Noise'' \\
out\_dark.fits           & MASTER\_DARK  & master dark \\
\hline
\end{longtable}

\subsubsection{Quality control}

Dark frames are processed to monitor the RON, Read Out Noise per DIT, 
the FPN, Fixed Patter Noise per DIT, the detector counts per DIT, 
the number of hot pixels per DIT.


%\begin{verbatim}
%QC.BP-MAP.METHOD           
%QC.BP-MAP.NBADPIX    
%QC.DARKMED.AVE        
%QC.DARKMED.STDEV    
%QC.RON         
%QC.RONRMS       
%QC.DARKFPN         
%QC.RON1            
%QC.RON2         
%\end{verbatim}

\paragraph{\bf RON}
The RON is computed on the whole detector chip and given as value of the QC.RON
parameter. For quality control those values are monitored as a 
function of time and DIT.
Two consecutive frames are subtracted from each other and the median 
standard deviation of a limited number of samples is taken and normalised to
DET.NDIT=1. The RON is computed in two regions and is given by the values of
QC.RON1 and QC.RON2.

\paragraph{Dark median counts}
The median and standard deviation of the counts in the master dark frame are
monitored by DFO. Its value and standard deviation are given by the values 
of QC.DARKMED.AVE and \\
QC.DARKMED.STDEV

\paragraph{Fixed Pattern Noise}
A histogram of the master dark is produced, and a fit is applied;
the standard deviation (sigma) of the Gaussian is the FPN. 
This value, logged by parameter QC.DARKFPN, is monitored for different 
DITs. The FPN should scale linearly with the number of counts. For this reason
the ratio FPN/counts is monitored for different DITs.

\paragraph{Number of hot pixels}
The number of pixels having an intensity greater than a threshold is monitored
in the parameter QC.BP-MAP.NBADPIX


\subsubsection{Parameters}


\begin{longtable}{|*3{l|}}
\hline
parameter                               & alias                      & default \\
\hline
sinfoni.bp\_noise.thresh\_sigma\_factor & bp\_noise-thresh\_sigma\_fct &  10.0 \\
sinfoni.bp\_noise.low\_rejection        & bp\_noise-lo\_rej           &  10.0 \\
sinfoni.bp\_noise.high\_rejection       & bp\_noise-hi\_rej           &  10.0 \\
sinfoni.dark.low\_rejection             & dark-lo\_rej                 &  0.1  \\
sinfoni.dark.high\_rejection            & dark-hi\_rej                 &  0.1  \\
sinfoni.dark.qc\_ron\_xmin              & dark-qc\_ron\_xmin           &  1    \\
sinfoni.dark.qc\_ron\_xmax              & dark-qc\_ron\_xmax           &  2048 \\
sinfoni.dark.qc\_ron\_ymin              & dark-qc\_ron\_ymin           &  1    \\
sinfoni.dark.qc\_ron\_ymax              & dark-qc\_ron\_ymax           &  2048 \\
sinfoni.dark.qc\_ron\_hsize             & dark-qc\_ron\_hsize          &  4    \\
sinfoni.dark.qc\_ron\_nsamp             & dark-qc\_ron\_nsamp          &  100  \\
sinfoni.dark.qc\_fpn\_xmin              & dark-qc\_fpn\_xmin           &  1  \\
sinfoni.dark.qc\_fpn\_xmax              & dark-qc\_fpn\_xmax           &  2047 \\
sinfoni.dark.qc\_fpn\_ymin              & dark-qc\_fpn\_ymin           &  1 \\
sinfoni.dark.qc\_fpn\_ymax              & dark-qc\_fpn\_ymax           &  2047 \\
sinfoni.dark.qc\_fpn\_hsize             & dark-qc\_ron\_hsize          &  2    \\
sinfoni.dark.qc\_fpn\_nsamp             & dark-qc\_ron\_nsamp          &  1000 \\
\hline
\end{longtable}

